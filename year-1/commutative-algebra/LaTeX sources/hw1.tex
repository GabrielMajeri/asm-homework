\section*{Problem set 1}
\stepcounter{section}

\begin{problem}
If \(I\) is an ideal in the ring \(R\), and \(p_1, \dots, p_n \in \Spec(R)\) for some \(n \in \naturals^*\) so that \(I \subseteq \bigcup_{i = 1}^{n} p_i\), then \(\exists k \in \overline{1, n}\) so that \(I \subseteq p_k\).
\end{problem}
\begin{proof}
This proof is based on the one presented by Robert B. Ash in \href{https://faculty.math.illinois.edu/~r-ash/ComAlg/ComAlg0.pdf}{the introduction to his commutative algebra book}, with additional explanations.

We will prove the statement above using induction:
\begin{itemize}
    \item If \(n = 1\), we have \(I \subseteq p_1\), and the statement is true for \(k = 1\).
    \item If \(n = 2\), we know that \(I \subseteq (p_1 \cup p_2)\).
    
    Suppose that \(I \not\subseteq p_1\) and \(I \not\subseteq p_2\). Then there exists at least one \(x_1 \in I\) which is also in \(p_1\) but not in \(p_2\), and at least one \(x_2 \in I\) which is also in \(p_2\) but not in \(p_1\) (the set differences \(I \setminus p_1\) and \(I \setminus p_2\) are not empty).
    
    We have \(x_1 + x_2 \in I\), since \(x_1, x_2 \in I\). However, \(x_1 + x_2\) isn't in \(p_1\), since \(x_1 \in p_1, x_2 \not\in p_1\) (and \(p_1\) is an ideal, and therefore closed under addition). It is neither in \(p_2\), since \(x_1 \not\in p_2, x_2 \in p_2\).
    
    Therefore \(x_1 + x_2 \not\in (p_1 \cup p_2) \implies x_1 + x_2 \not\in I\), a contradiction.

    \item For \(n \geq 3\), we reason similarly to the case above.
    
    Suppose that \(I \not\subseteq p_1, I \not\subseteq p_2, \dots, I \not\subseteq p_n\). Additionally, we can assume that \(I\) is not included in any union of \(k < n\) of the prime ideals, since that means we can ignore the other \(n - k\) ideals and just apply the induction hypothesis for a set of \(k\) prime ideals.
    
    In particular, for \(k = n - 1\), this tells us we can take \(x_1, x_2, \dots, x_n \in I\), where each \(x_i\) belongs to \(I \setminus \left(\bigcup_{\substack{j = 1 \\ j \neq i}}^{n} p_j\right)\).
    
    We consider the product \(x' = x_1 x_2 \dots x_{n-1} \in I\). Because of the way the \(x_i\) were defined, and because \(p_n\) is prime, \(x' \not\in p_n\). Additionally, \(x' \in p_1 \cdot \dots \cdot p_{n-1} \subseteq p_1 \cap \dots \cap p_{n-1} \subseteq p_1 \cup \dots \cup p_{n-1}\).
    
    Now look at \(x' + x_n\):
    \begin{itemize}
        \item It cannot belong to any \(p_i\) with \(i < n\), since then \((x' + x_n) - x' \in p_i \implies x_n \in p_i\) (a contradiction with the way \(x_n\) was defined). This can be restated as \(x' + x_n \not\in p_1 \cup \dots \cup p_{n - 1}\).
        \item It cannot belong to \(p_n\) either, since then \((x' + x_n) - x_n \in p_n \implies x' \in p_n)\).
    \end{itemize} Therefore \(x' + x_n \not\in \bigcup_{i = 1}^{n} p_i\), a contradiction with the initial hypothesis.
\end{itemize}
\end{proof}

\begin{problem}
Decide if \((3, X^3 - X^2 + 2X + 1) \in \Spec(\integers[X])\).
\end{problem}
\begin{proof}
We rely on the fact that \(I \text{ prime} \iff R/I \text{ integral domain}\).

First, \(3\) is prime so we know that
\[
\frac{\integers[X]}{(3, X^3 - X^2 + 2X + 1)} \cong \frac{\integers_3[X]}{(\widehat{X^3} - \widehat{X^2} + \widehat{2}\widehat{X} + \widehat{1})}
\]

In \(\integers_3[X]\), we have
\[
    \widehat{X^3} - \widehat{X^2} + \widehat{2}\widehat{X} + \widehat{1} = \widehat{X^3} + \widehat{2}\widehat{X^2} + \widehat{2}\widehat{X} + \widehat{1} = (\widehat{X} + \widehat{1})(\widehat{X^2} + \widehat{X} + \widehat{1})
\]

This shows that \(\integers_3[X] / (\widehat{X^3} - \widehat{X^2} + \widehat{2}\widehat{X} + \widehat{1})\) is not an integral domain, since the classes of \(X + 1\) and \(X^2 + X + 1\) would be zero divisors. 

We conclude that \((3, X^3 - X^2 + 2X + 1)\) isn't prime.
\end{proof}

\begin{problem}\hfill
\begin{enumerate}[(i)]
    \item Decide if \((X^4 - Y^6) \in \Spec(\reals[X, Y])\).
    \item Decide if \((X^5 - Y^7) \in \Spec(\reals[X, Y])\).
\end{enumerate}
\end{problem}
\begin{proof}\hfill
\begin{enumerate}[(i)]
    \item We can factor \(X^4 - Y^6\) as \((X^2 - Y^3)(X^2 + Y^3)\).
    
    Therefore, \(\reals[X, Y]/(X^4-Y^6)\) isn't an integral domain, since it has zero divisors (for example, the classes of \(X^2 - Y^3\) and \(X^2 + Y^3\)).
    
    Using the same argument as in the previous problem, we conclude that the ideal \((X^4 - Y^6)\) isn't prime.
    
    \item We are going to use the lemma which states that for any ring homomorphism \(f \colon R \to S\), if \(q\) is a prime ideal of \(S\) then \(f^{-1}(q)\) is a prime ideal of \(R\).
    
    In our case, we will use the unique \(\reals\)-algebra evaluation homomorphism \(\phi \colon \reals[X, Y] \to \reals[t]\), given by \(\phi(X) = t^7\) and \(\phi(Y) = t^5\). \(\phi\) can also be seen as a ring homomorphism.
    
    Note that \(\set{0} \in \Spec(\reals[t])\), and we will determine \(\phi^{-1}(\set{0}) = \ker \phi \in \Spec(\reals[X, Y])\).

    To determine \(\ker \phi\) we will use a method analogous to the one described \href{https://math.stackexchange.com/a/488437/388180}{here}.
    First, we observe that \(X^5 - Y^7 \in \ker \phi\). For any \(g \in \ker \phi\), we can use polynomial division in X to write:
    \[g(X, Y) = (X^5 - Y^7) f_1(X, Y) + X^4 f_2 (Y) + X^3 f_3 (Y) + X^2 f_4 (Y) + X f_5 (Y) + r(Y)\]
    
    Since \(\ker \phi\) is an ideal, and \(g\) and \(X^5 - Y^7\) are in it, we get that \(X^4 f_2 (Y) + X^3 f_3 (Y) + X^2 f_4 (Y) + X f_5 (Y) + r(Y)\) is also in \(\ker \phi\).
    
    Applying \(\phi\) to this expression, we have:
    \begin{gather*}
        \phi(X^4 f_2 (Y) + X^3 f_3 (Y) + X^2 f_4 (Y) + X f_5 (Y) + r(Y)) = 0 \iff \\
        \phi(X^4) f_2(\phi(Y)) + \phi(X^3) f_3(\phi(Y)) + \phi(X^2) f_4(\phi(Y)) \\
        + \phi(X) f_5(\phi(Y)) + r(\phi(Y)) = 0 \iff \\
        t^{28} f_2(t^5) + t^{21} f_3(t^5) + t^{14} f_4(t^5) + t^7 f_5(t^5) + r(t^5) = 0
    \end{gather*}
    
    All of the exponents of the first term will be \(28 + 5 \cdot k\), the exponents of the second term will be \(21 + 5 \cdot k'\), and so on. Notice that \(0, 7, 14, 21, 28\) are all different modulo 5. The only way for them to match is if \(k = k' = \dots = 0\). Therefore, \(f_2, f_3, \dots, r \in \reals\), and they must be \(0\). 
    
    This means \(\ker \phi\) is precisely the multiples of \(X^5 - Y^7\), i.e. \((X^5 - Y^7)\). We get that \(\ker \phi = (X^5 - Y^7)\) is a prime ideal in \(\reals[X, Y]\).
\end{enumerate}
\end{proof}

\begin{problem}
Let \(p \in \Spec(R)\).
\begin{enumerate}[(i)]
    \item Show that the ideal \(q \coloneqq (p, X) R[X]\) is prime.
    \item Show that the ideal \(p^{\symcal{L}} \coloneqq p R[X]\) is prime.
    \item Show that \(p^{\symcal{L}} \subsetneq q\) and that \(p^{\symcal{L}} \cap R = q \cap R = p\).
\end{enumerate}
\end{problem}
\begin{proof}\hfill
\begin{enumerate}[(i)]
    \item Take \(f, g \in R[X]\) such that \(f \cdot g \in q\). We want to show that \(f \in q\) or \(g \in q\).
    
    We can write out that the product of \(f\) and \(g\) is in \(q\) as:
    \[
        (a_n X^n + \dots + a_0) \cdot (b_m X^m + \dots + b_0) = r \cdot h_1 + X \cdot h_2
    \]
    for some \(r \in p\) and \(h_1, h_2 \in R[X]\).
    
    If \(f\) (or \(g\)) has a zero constant term, then we're done, since \(f\) (or \(g\)) would be divisible by \(X\).
    
    We will consider the equality of the constant terms in the equation above. We can ignore the \(X \cdot h_2\) term on the right hand side, since it has no (non-zero) constant term.
    
    The constant term on the left-hand side is \(a_0 \cdot b_0\), and if we label the constant term of \(h_1\) with \(c_0\), the constant term on the right-hand side is \(r \cdot c_0 \in p\).
    
    We have that \(a_0 \cdot b_0 \in p\), therefore \(a_0 \in p\) or \(b_0 \in p\) (since \(p\) is prime). But that means \(f\) or \(g\) are a sum of an element of \(p\) and a multiple of \(X\), meaning \(f \in q\) or \(g \in q\).
    
    \item We know that \(R[X]/I \cong (R/I)[X]\) for any \(I \trianglelefteq R\). In particular for our case, \(R[X]/(p) \cong (R/(p))[X]\). We know that the right hand side is an integral domain, since \(R / (p)\) is an integral domain (\(p\) is prime). Therefore, \(R[X]/(p)\) is an integral domain, and this means that \((p)R[X] = p^{\symcal{L}}\) is prime.
    
    \item It's easy to see that \(p^{\symcal{L}} \subset q\) since \((p) \subset (p, X)\). To show that the inclusion is strict, take any \(r \in p\) and consider \(r + X\). This is an element of \(q\) since it can be written as \(1 \cdot r + 1 \cdot X\). However, it is not divisible by any non-unit scalar in \(R\), therefore it's not in \(R \, R[X]\) and certainly not in \(p \, R[X] = p^{\symcal{L}}\).
\end{enumerate}
\end{proof}