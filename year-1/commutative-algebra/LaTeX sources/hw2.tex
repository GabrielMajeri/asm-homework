\section*{Problem set 2}
\stepcounter{section}

\begin{problem*}{1.2 from Kemper's book}
\newcommand{\KX}{\powerseriesring{K}{X}}

Consider the formal power series ring
\[\KX \coloneqq \Set{ \sum_{i=0}^{\infty} a_i x^i \mid a_i \in K }\]
over a field \(K\).

\begin{enumerate}[(a)]
    \item Show that \(\KX\) is an integral domain.
    \begin{proof}
        Suppose that \(\KX\) is not an integral domain, therefore it has zero divisors. Take \(f, g \in \KX\) such that \(f \neq 0\), \(g \neq 0\) but \(f g = 0\). If \(a_i\), \(b_i\) are the coefficients of \(f\), \(g\) then their product can be written out as
        \[
            fg = \sum_{i = 0}^{\infty}
            \left(
                \sum_{k = 0}^{i} a_k b_{i - k}
            \right)
        \]
        Take \(n\) to be the index of the first non-zero term of \(f\), and \(m\) the index of the first non-zero term of \(g\) (these must exist since \(f \neq 0\), \(g \neq 0\)). For \(i = n + m\), the sum is non-zero since \(a_{k} b_{i - k} = 0\) when \(k \neq n\), but \(a_{n} b_{m} \neq 0\). Therefore, we get a contradiction with \(f g = 0\).
    \end{proof}
    
    \item Show that all power series \(f = \sum_{i = 0}^{\infty} a_i x^i\) with \(a_0 \neq 0\) are invertible in \(\KX\). Assume for a moment that \(K\) is only a ring, show that \(f\) is invertible if and only if \(a_0\) is invertible in \(K\).
    \begin{proof}
    The identity element in \(\KX\) is the formal power series where only the first coefficient is \(1\) and all the rest are \(0\) (this way, all of the sums defining the coefficients of the product \(\sum_{k = 0}^{i} a_k b_i\) equal \(a_k\)).
    
    We will construct a new formal power series \(f^{-1}\) with coefficients \(b_i\) so that \(f \cdot f^{-1} = 1\). Since \(a_0 \neq 0\) is invertible in \(K\), define \(b_0 = a_0^{-1}\).
    
    Then, for each \(i \in \naturals^*\), let
    \[
        b_i = -\frac{\sum_{k = 1}^{i} a_k b_{i-k}}{a_0}
    \]
    These can be computed sequentially, since each \(b_i\) depends only on \(b_{0}, \dots, b_{i-1}\). Rewriting the expression above we get that
    \[
        a_0 b_i = - \sum_{k = 1}^{i} a_k b_{i-k} \iff \sum_{k=0}^{i} a_k b_{i-k} = 0, \forall i \in \naturals^*
    \]
    which proves that \(f^{-1}\) as defined above is indeed the inverse of \(f\).
    
    For the second part of the problem: if \(a_0\) is invertible, then the inverse can be constructed as above. Otherwise, if such a \(g\) would exist, then \(a_0 b_0 = 1\), contradicting the assumption that \(a_0\) has no inverse in \(K\).
    \end{proof}
    
    \item Show that \(\KX\) has exactly one maximal ideal \(m\).
    \begin{proof}
    Using a result from the previous homework, the statement to be proven is equivalent with the fact that \(\KX \setminus U(\KX)\) is a subgroup with respect to addition.
    
    Take \(f, g \in \KX\) such that both are non-units. Let \(a_i\) and \(b_i\) be their coefficients. By the previous exercise, we know that an element of \(\KX\) is invertible if and only if their constant term is invertible, therefore in our case \(a_0 = b_0 = 0\).
    
    The constant term of \(f - g\) is \(0 - 0 = 0\), therefore \(f - g \in \KX\setminus(\KX)\). But this is enough to prove that \(\KX\setminus(\KX)\) is a subgroup.
    \end{proof}
    
    \item Show that \(\KX\) is not a Jacobson ring.
    \begin{proof}
    Suppose \(\KX\) is a Jacobson ring. Then, for every proper ideal \(I\) of \(\KX\) we must have, by definition:
    \[
        \sqrt{I} = \bigcap_{\substack{m \in \Spec_{Max}(R) \\ I \subseteq m}} m
    \]
    In our case, this can be written as:
    \[
        \sqrt{I} = m
    \]
    where \(m\) is the unique maximal ideal \(m = (x)\).
    
    Consider \(I = (x + x^2)\) (that is, the ideal generated by the formal power series with \(a_1 = a_2 = 1\), and \(a_i = 0\) for all other \(i\)). We have \(\sqrt{I} = I\). But \(x \in m\), yet \(x \not\in I = \sqrt{I} \implies m \not\subseteq \sqrt{I}\), so \(m \neq \sqrt{I}\).
    \end{proof}
    
    \item Show that the ring
    \[
        L \coloneqq \Set{ \sum_{i = k}^{\infty} a_i x^i \mid k \in \integers, a_i \in K }
    \]
    of formal Laurent series is a field.
    \begin{proof}
    We can construct the inverse of any formal Laurent series \(f \neq 0\) as follows:
    \begin{itemize}
        \item If \(a_i = 0\) for all \(i < 0\), then \(f\) can also be seen as a formal power series. If \(a_0 \neq 0\), then we can compute the inverse as we described previously. Otherwise, let \(k\) be the smallest positive integer for which \(a_k \neq 0\) (it exists since \(f\) is non-zero). We can multiply \(f\) by \(x^{-k}\), obtaining an invertible formal power series, for which we can compute the inverse as before, and then multiply this by \(x^k\), to get the inverse of \(f\).
        \item Let \(k\) be the smallest negative integer for which \(a_k \neq 0\). By multiplying \(f\) with \(x^k\), we get a formal power series with a non-zero constant term, for which we can compute the inverse as above, then multiply with \(x^{-k}\) to get the inverse for the initial formal Laurent series.
    \end{itemize}
    \end{proof}
    
    \item Is \(\KX\) finitely generated as a \(K\)-algebra?
    \begin{proof}
    If \(\KX\) were finitely generated, then it would be an affine \(K\)-algebra. Then, by Theorem 1.13 from Kemper's book, we get that \(\KX\) is a Jacobson ring. But we've just proven the opposite in the previous exercise. 
    \end{proof}
\end{enumerate}
\end{problem*}

\begin{problem*}{1.7 from Kemper's Book}
Consider the ideal
\[
    I = (x_1^4 + x_2^4 + 2 x_1^2 x_2^2 - x_1^2 - x_2^2) \subseteq \reals[x_1, x_2].
\]
\begin{enumerate}[(a)]
    \item Determine \(X \coloneqq \symcal{V}(I) \subseteq \reals^2\) and draw a picture.
    \begin{proof}
    Let \(X \coloneqq x_1^2\), \(Y \coloneqq x_2^2\). Then the polynomial above factors as:
    \begin{align*}
        &\phantom{=} X^2 + Y^2 + 2 X Y - X - Y \\
        &= (X + Y)^2 - X - Y \\
        &= (X + Y)(X + Y) - (X + Y) \\
        &= (X + Y)(X + Y - 1)
    \end{align*}
    Undoing the substitution yields:
    \[
        (x_1^2 + x_2^2) (x_1^2 + x_2^2 - 1)
    \]
    If the generator polynomial of the ideal vanishes on a point \(\xi \in \reals^2\), then any multiple of it will also vanish. Therefore, to determine \(\symcal{V}(I)\), it's enough to find the zeroes of the generator polynomial.

    Setting \(x_1^2 + x_2^2 = 0\) gives us \((0, 0) \in \reals^2\), and \(x_1^2 + x_2^2 - 1 = 0 \iff x_1^2 + x_2^2 = 1\) is the unit circle. The graphical representation of \(X\) is in figure \ref{problem1.7}.
    
    \begin{figure}[htbp]
        \centering
        \includegraphics[width=0.5\textwidth]{hw2-p1.7.png}
        \caption{Image generated using \href{https://www.geogebra.org/calculator}{Geogebra}.}
        \label{problem1.7}
    \end{figure}
    \end{proof}
    
    \item Is \(I\) a prime ideal? Is \(I\) a radical ideal?
    \begin{proof}
    \(I\) is not a prime ideal, since \((x_1^2 + x_2^2)(x_1^2 + x_2^2 - 1) \in I\), but neither term of the product belongs to \(I\).
    
    \(I\) is a radical ideal iff \(I = \sqrt{I}\). To compute \(\sqrt{I}\), it's enough to intersect all the prime ideals containing \(I\). Since \(x_1^2 + x_2^2\) and \(x_1^2 + x_2^2 - 1\) are irreducible over \(\reals\) and \(\reals[x_1, x_2]\) is a unique factorization domain, the ideals they generate are prime.
    \[
        \sqrt{I} = (x_1^2 + x_2^2) \cap (x_1^2 + x_2^2 - 1) = ((x_1^2 + x_2^2)(x_1^2 + x_2^2 - 1)) = I
    \]
    Hence \(I\) is a radical ideal.
    \end{proof}
    
    \item Does Hilbert's Nullstellensatz hold for \(I\)?
    \begin{proof}
    To check the theorem, we need to determine \(\symcal{I}(\symcal{V}(I))\). That is, we need to find the ideal generated by the polynomials \(f \in \reals[x_1, x_2]\) for which \(f(0, 0) = 0\) and \(f(x_1, x_2) = 0\) for all \(x_1, x_2\) such that \(x_1^2 + x_2^2 = 1\).
    
    The first equality tells us \(f\) has a zero constant term, while the second one tells us \(f\) vanishes on the unit circle. From the first equality, we get that \(f\) should be a multiple of some \(x^{2k} + y^{2k}\) (since these are the only polynomials in \(\reals[x_1, x_2]\) whose only root is \((0, 0)\)) and a multiple of \(x^2 + y^2 - 1\). Because \(\symcal{I}(\symcal{V}(I))\) should contain all of the polynomials which vanish at those points, we take the least restrictive \(k = 1\). Then \(\symcal{I}(\symcal{V}(I)) = (x_1^2 + x_2^2) \cap (x_1^2 + x_2^2 - 1) = I\).
    \end{proof}
\end{enumerate}
\end{problem*}

\begin{problem*}{6}
Let \(I\) be a monomial ideal in \(K[X_1, \dots, X_n]\).
\begin{enumerate}[(i)]
    \item Show that if \(0 \neq f \in I\) and \(m\) is any monomial term of \(f\), then \(m \in I\).
    \begin{proof}
    Let \(I = (m_1, \dots, m_n)\) for some \(n \in \naturals\). Then \(f \in I\) means that
    \[
        f = g_1 m_1 + \dots + g_n m_n
    \]
    for some \(g_1, \dots, g_n \in K[X_1, \dots, X_n]\).
    
    If we expand each \(g_i\) above as a product of monomials, we get:
    \begin{align*}
        f &= (m^{(g_1)}_1 + \dots + m^{(g_1)}_{k_{g_1}}) m_1 + \dots + (m^{(g_n)}_1 + \dots + m^{(g_n)}_{k_{g_n}}) m_n \\
        &= m_1 m^{(g_1)}_1 + \dots + m_1 m^{(g_1)}_{k_{g_1}} + \dots + m_n m^{(g_n)}_1 + \dots + m_n m^{(g_n)}_{k_{g_n}}
    \end{align*}
    So each monomial term \(m\) of \(f\) is a multiple of a monomial \(m_i\) from \(I\), therefore \(m \in I\).
    \end{proof}
    
    \item Show that \(I\) has a unique minimal generating set consisting of monomials. This set will be denoted \(G(I)\).
    \begin{proof}
    We can construct such a set recursively as follows:
    \begin{itemize}
        \item Let \(G_1(I) = \Set{ m_1 }\), where \(m_1\) is any monomial in \(I\).
        \item Take any other monomial \(m_i \in I\) such that \(m_i \not\in (G_{i-1}(I))\) and let \(G_{i}(I) = G_{i-1}(I) \cup \Set{ m_i }\).

        If there is no such \(m_i\), then \((G_{i-1}(I)) = I\). Let \(G(I) = G_{i-1}(I)\) and we are done.
    \end{itemize}
    The algorithm described above always terminates in a finite number of steps, since otherwise \((G_1(I)) \subseteq (G_2(I)) \subseteq \dots\) would be an ascending sequence of ideals which doesn't stabilize, but we know that \(K[X_1, \dots, X_n]\) is Noetherian.
    
    \(G(I)\) as constructed above is minimal and unique. Suppose we had a (possibly smaller) different set \((G'(I)) = (m'_1, m'_2, \dots, m'_k)\) which could generate all of \(I\). Then \((G(I)) = (m_1, m_2, \dots, m_n)\) would be a generator for \(f = m'_1 + \dots + m'_k\). Based on the previous exercise, this would mean \(\forall j \in \overline{1, k}, \exists i \in \overline{1, n}\) such that \(m_i \mid m'_j\). Since \(m_i, m'_j\) are monomials, if \(k = n\) then \(G'(I) = G(I)\), otherwise if \(k < n\) there is at least one monomial \(m_h \in G(I)\) which cannot be written as a linear combination of \(m'_1, \dots, m'_k\) (because it would be a linear combination of the corresponding \(m_i\)).
    \end{proof}
\end{enumerate}
\end{problem*}

