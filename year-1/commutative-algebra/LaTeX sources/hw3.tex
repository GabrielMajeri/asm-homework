\section*{Problem set 3}
\stepcounter{section}

\begin{problem}
Let \(p\) be a prime integer. We denote
\[
    H = \Set{ \frac{a}{p^n} \mid a \in \integers, n \in \integers }
\]
\begin{enumerate}[(i)]
    \item Show that \(H/\integers\) is an Artinian \(\integers\)-module which is not Noetherian.
    \begin{proof}
    Let \(M\) be a \(\integers\)-submodule of \(H / \integers\).
    
    We remark that:
    \begin{itemize}
        \item If \(\widehat{\frac{a}{p^n}} \in M\) then \(p \widehat{\frac{a}{p^n}} = \widehat{\frac{p a}{p^n}} = \widehat{\frac{a}{p^{n - 1}}} \in M\).
        
        \item  If \(a\) and \(p^n\) are coprime, then by Euler's theorem we have
        \[
        a^{\phi(p^n) - 1} \widehat{\frac{a}{p^n}} = \widehat{\frac{a^{\phi(p^n)}}{p^n}} = \widehat{\frac{k p^n + 1}{p^n}} = \widehat{k} + \widehat{\frac{1}{p^n}} = \widehat{0} + \widehat{\frac{1}{p^n}} = \widehat{\frac{1}{p^n}}
        \]
        (where \(\phi\) represents Euler's totient function).
    \end{itemize}
    
    These observations allow us to conclude that any proper \(\integers\)-submodule of \(H/\integers\) is generated by \(\widehat{\frac{1}{p^n}}\) for some \(n \in \integers\).
    
    Consider any descending chain of submodules
    \[M_1 \supseteq M_2 \supseteq \dots\]

    We can assume \(M_1 \neq H/\integers\) (otherwise, start from the first \(M_i\) which is not the whole module). Write \(M_1 = \widehat{\frac{1}{p^{k_1}}}\) for some \(k_1 \in \integers\). If \(k_1 \leq 0\), \(M_1 = \integers = 0\), and the chain stabilizes.
    
    Let \(k_i\) be the largest integer such that \(\widehat{\frac{1}{p^{k_i}}} \in M_i\). If \(M_i \supsetneq M_{i + 1}\), then \(\widehat{\frac{1}{p^{k_i}}} \not\in M_{i + 1}\). Clearly, we have \(k_{i + 1} < k_{i} < \dots k_1\). Since \(k_1\) is a finite integer, there must be some \(j\) where \(k_j = 0\), i.e. \(M_j = \left(\widehat{\frac{1}{p^0}}\right) = \left(\widehat{1}\right) = 0\). Then \(0 = M_j = M_{j + 1} = \dots\), proving that \(H/\integers\) is Artinian.
    
    As a counterexample for Noetherianity, consider the ascending chain of submodules
    \[
        \left(\widehat{\frac{1}{p}}\right) \subsetneq \left(\widehat{\frac{1}{p^2}}\right) \subsetneq \dots \subsetneq \left(\widehat{\frac{1}{p^n}}\right) \subsetneq \dots
    \]
    which doesn't stabilize.
    \end{proof}
    
    \item Show that \(H\) is neither an Artinian, nor a Noetherian \(\integers\)-module.
    \begin{proof}
    For \(n = 0\), \(\frac{a}{p^0} = a\), \(\forall a \in \integers\). Therefore \(\integers\) is contained in \(H\).
    
    Consider the descending chain of submodules
    \[
        (2)\integers \supsetneq (4)\integers \supsetneq (8)\integers \supsetneq \dots
    \]
    which doesn't terminate, therefore proving \(H\) is not Artinian.
    
    Using the same chain as above, we can show \(H\) is not Noetherian:
    \[
        \left(\frac{1}{p}\right) \subsetneq \left(\frac{1}{p^2}\right) \subsetneq \dots \subsetneq \left(\frac{1}{p^n}\right) \subsetneq \dots
    \]
    \end{proof}
\end{enumerate}
\end{problem}

\begin{problem}[Cohen]
Prove that the ring \(R\) is Noetherian iff any prime ideal in \(R\) is finitely generated.
\end{problem}
\begin{proof}
The proof of the theorem is based on the one found in the book ``Steps in Commutative Algebra'' by R. Y. Sharp.

First we prove the direct implication. If \(R\) is Noetherian, then every ideal is finitely generated. In particular, every prime ideal is finitely generated.

To prove the converse, assume that the ring \(R\) is not Noetherian. Let \(S\) be the set
\[
    S = \Set{ I \leq R \mid I \text{ is not finitely generated} }
\]
It is not empty, since otherwise every ideal would be finitely generated, a contradiction with our assumption that the ring is not Noetherian.

Consider a chain of ideals contained in \(S\)
\[
    I_1 \subseteq I_2 \subseteq \dots
\]
Then \(U = \bigcup_{i = 1}^{\infty} I_i\) is also in \(S\), since if it were finitely generated by some elements \(r_1, \dots, r_k\), with \(r_j \in I_{i_j}\), there would be some \(I_n\) containing \(\bigcup_{j = 1}^{k} I_{i_j}\), contradicting the fact that no \(I_i\) is finitely generated. Thus, \(U\) is an upper bound for the chain.

We can now apply Zorn's lemma and conclude that \(S\) must have a maximal element, which we will label \(J\). It cannot be a prime ideal, since all prime ideals in \(R\) are finitely generated, and \(J \in S\). Therefore, \(\exists x, y \in R\) such that \(x y \in J\) but \(x \not\in J\), \(y \not\in J\).

Let us consider the ideal \(J + xR\). Clearly, \(J \subsetneq J + xR\). This means \(J + xR\) must be finitely generated, since \(J\) is maximal in \(S\). We will label the generating set of \(J + xR\) as \(f_i = a_i + x b_i\), with \(a_i \in J\) and \(b_i \in R\), \(\forall i \in \overline{1, n}\).

Take \(z \in J\). Since \(z\) is also in \(J + xR\) and \(J + xR\) is finitely generated, we can write it as:
\begin{align*}
    z &= \sum_{i = 1}^{n} r_i \cdot \underbrace{f_i}_{\in J + xR} \\
    &= \sum_{i = 1}^{n} r_i \cdot (\underbrace{a_i}_{\in J} + \, x \underbrace{b_i}_{\in R}) \\
    &= \sum_{i = 1}^{n} \underbrace{r_i a_i}_{\in J} + \, x \sum_{i = 1}^{n} \underbrace{r_i b_i}_{\in R}
\end{align*}
Since \(z \in J\) and \(r_i a_i \in J, \forall i \in \overline{1, n}\), we have that \(x \cdot r_i b_i \in J, \forall i \in \overline{1, n}\). But the elements \(r_i b_i\) are contained in \((J : x) = \Set{ r \in R \mid rx \in J}\). We can rewrite the equality above as the following inclusion:
\[
    J \subseteq \sum_{i = 1}^{n} R a_i + x \cdot (J : x)
\]
Furthermore, since \(x \cdot (J : x) \subseteq J\), the reverse inclusion also holds, meaning we have equality:
\[
    J = \sum_{i = 1}^{n} R a_i + x \cdot (J : x)
\]
From the definition of the colon ideal we get that \(J \subseteq (J : x)\). Notice that \(y \in (J : x)\) but \(y \not\in J\) and \(y \not \in (x)\), therefore \(J \neq (J : x)\). By the maximality of \(J\) in \(S\), we conclude that \((J : x)\) must be finitely generated.

Let \(c_i \in (J : x), \, i \in \overline{1, m}\) be the generators of \((J : x)\). Then we have that
\[
    J = \sum_{i = 1}^{n} R a_i + x \sum_{i = 1}^{m} R c_i
\]
showing that \(J\) is finitely generated. This contradicts \(J \in S\), showing that \(J\) must be prime. But by the hypothesis this means \(J\) is finitely generated, therefore \(S\) must be the empty set and our assumption was false. \(R\) is Noetherian.
\end{proof}

\begin{definition}
Recall that for \(I, J\) ideals in \(R\), their \textbf{colon ideal} is defined as
\[
    I : J = \Set{ a \in R \mid a \cdot J \subseteq I }
\]
\end{definition}

\begin{problem*}{5}
If \(R\) is any ring and \(I\), \(J\) and \(U\) are any ideals of it, then:
\begin{enumerate}[(i)]
    \item \(I : J\) is an ideal of \(R\);
    \begin{proof}
    Let \(r_1, r_2 \in (I : J)\). This means \(r_1 x \in I\), \(r_2 x \in I\), \(\forall x \in J\). Then \(r_1 x - r_2 x \in J \implies ((r_1 - r_2) x \in J) \implies (r_1 - r_2) \in (I : J)\), as needed.
    
    By the same logic, \((r_1 r_2) x = r_1 (r_2 x) = r_1 y\), \(\forall x \in J\), where \(y = r_2 x \in I\), therefore \(r_1 y \in I \implies (r_1 r_2) x \in I\).
    \end{proof}
    
    \item \(I \subseteq I : J\);
    \begin{proof}
    Let \(y \in I\). Then \(y x \in I\), \(\forall x \in J\) (due to the absorbing property of ideals), therefore \(y \in (I : J)\).
    \end{proof}
    
    \item \((I : J) : U = I : (J \cdot U) = (I : U) : J\);
    \begin{proof}
    \begin{align*}
        (I : J) : U &= \Set{ r \in R \mid r U \subseteq (I : J) } \tag{expand definition of \((I : J) : U\)} \\
        &= \Set{ r_1 \in R \mid r_1 U \subseteq \Set{ r_2 \in R \mid r_2 J \subseteq I } } \tag{expand definition of \(I : J\)} \\
        &= \Set{ r_1 \in R \mid \forall u \in U, r_1 u = r_2, r_2 J \subseteq I } \tag{rewrite inclusion pointwise} \\
        &= \Set { r \in R \mid rUJ \subseteq I } \tag{write as multiplication of ideals} \\
        &= (I : (U \cdot J)) \tag{by definition of \(I : (U \cdot J)\)} \\
        &= \Set { r \in R \mid rJU \subseteq I } \tag{by commutativity} \\
        &= (I : (J \cdot U)) \tag{by definition of \(I : (J \cdot U)\)} \\
        &= (I : J) : U \tag{analogous to steps above}
    \end{align*}
    \end{proof}
    
    \item \(\left(\bigcap_{i \in \Lambda} I_i\right) : J = \bigcap_{i \in \Lambda} \left(I_i : J\right)\)
    \begin{proof}
    Let \(x \in R\).
    \begin{gather*}
        x \in \left(\bigcap_{i \in \Lambda} I_i\right) : J \iff \\
        x J \subseteq \bigcap_{i \in \Lambda} I_i \iff \\
        x J \subseteq I_i, \forall i \in \Lambda \iff \\
        x \in I_i : J, \forall i \in \Lambda \iff \\
        x \in \bigcap_{i \in \Lambda} (I_i : J)
    \end{gather*}
    \end{proof}
    
    \item \(I : \left(\sum_{i \in \Lambda} J_i\right) = \bigcap_{i \in \Lambda} \left(I : J_i\right)\)
    \begin{proof}
    Let \(x \in R\).
    \begin{gather*}
        x \in I : \left(\sum_{i \in \Lambda} J_i\right) \iff \\
        x \left(\sum_{i \in \Lambda} J_i\right) \subseteq I \implies \\
        x J_i \subseteq I, \forall i \in \Lambda \iff \\
        x \in I : J_i, \forall i \in \Lambda \iff \\
        x \in \bigcap_{i \in \Lambda} \left(I : J_i\right)
    \end{gather*}
    We actually have equivalence at the second step because
    \[
        x J_i \subseteq I, \forall i \in \Lambda
        \implies
        x \left(\bigcap_{i \in \Lambda} J_i \right) \subseteq I
        \implies
        x \left(\sum_{i \in \Lambda} J_i\right) \subseteq I
    \]
    (the intersection of ideals is contained within their sum).
    \end{proof}
\end{enumerate}
\end{problem*}

\begin{problem*}{6}
In the polynomial ring \(\integers[X]\) show that
\begin{enumerate}[(i)]
    \item the ideal \((9, X)\) is a primary ideal;
    \begin{proof}
    We begin by noting that both \(9\) and \(X\) are monomials in \(\integers[X]\), and a result from the course tells us that the radical of a monomial ideal can be computed as \(\sqrt{I} = \sqrt{(9, X)} = (3, X)\).
    
    Suppose that \(I\) isn't primary. Then \(\exists f, g \in \integers[X]\) such that \(f \not\in \sqrt{I}\), \(g \not \in I\) but \(f g \in I\). This means
    \[
        fg = h_1 \cdot 9 + h_2 \cdot X
    \]
    for some \(h_1, h_2 \in \integers[X]\).
    
    We will look only at the constant terms of each side of the equality. On the right side, the constant term is either a multiple of 9, or 0 (when \(h_1 = h_2 = 0\)). The latter case would imply \(fg = 0\), which cannot happen since \(\integers[X]\) is an integral domain.
    
    Since \(f \not\in (3, X)\), in particular \(f \not\in (3)\). So the constant term of \(f\) is not a multiple of \(3\).
    
    Since \(g \not\in (9, X)\), in particular \(g \not\in (9)\). So the constant term of \(f\) is not a multiple of \(9\).
    
    But now we get that the product of an integer which is not a multiple of 3 and an integer which is not a multiple of 9 is a multiple of 9, which is impossible. This contradicts the assumption that \(I\) is primary.
    \end{proof}
    
    \item the ideal \(J = (X^2, 2X)\) is not a primary ideal.
    \begin{proof}
    As above, the radical of \(J\) is \(\sqrt{J} = (X)\).
    
    Let \(f = 2\), \(g = X^2 + X\). Then \(f \not\in \sqrt{J}\), \(g \not\in J\) (since the coefficient of the \(X\) term in \(J\) is always even) yet
    \[
        2 \cdot (X^2 + X) = 2X^2 + 2X \in J
    \]
    which shows that \(J\) isn't primary.
    \end{proof}
\end{enumerate}
\end{problem*}

\begin{problem*}{7}
Let \(R = K[X_1, \dots, X_n]\) be the polynomial ring in \(n\) indeterminates, where \(K\) is a field. For any monomial ideal \(I \subset R\), let \(G(I)\) denote the unique minimal system of generators for \(I\) consisting only of monomials (see the previous homeworks).
\begin{enumerate}[(i)]
    \item Show that if \(u\) and \(v\) are coprime monomials, and \(g_1, \dots, g_l\) are monomials in \(R\), then
    \[
        (u \cdot v, g_1, \dots, g_l) = (u, g_1, \dots, g_l) \cap (v, g_1, \dots, g_l)
    \]
    \begin{proof}
    We will prove this by double inclusion.
    
    Let \(f \in (u \cdot v, g_1, \dots, g_l)\). Then
    \begin{align*}
        f &= f_0 uv + f_1 g_1 + \dots + f_l g_l \\
        &= (f_0 v) u + f_1 g_1 + \dots + f_l g_l \\
        &= (f_0 u) v + f_1 g_1 + \dots + f_l g_l
    \end{align*}
    which shows that \(f \in (u, g_1, \dots, g_l)\) and that \(f \in (v, g_1, \dots, g_l)\), respectively. Therefore \(f \in (u, g_1, \dots, g_l) \cap (v, g_1, \dots, g_l)\).
    
    Let \(f \in (u, g_1, \dots, g_l) \cap (v, g_1, \dots, g_l)\). This tells us that \(u \divides f\), \(v \divides f\), and \(g_i \divides f\) for each \(i\).
    
    Using the properties of the \(\lcm\), since \(u \divides f\) and \(v \divides f\) if follows that \(\lcm(u, v) \divides f\). But since \(u\) and \(v\) are coprime, \(\lcm(u, v) = u \cdot v\). Hence \(u \cdot v \divides f_0\). Then \(f \in (u \cdot v, g_1, \dots, g_l)\), as needed.
    \end{proof}
    
    \item Prove that a monomial ideal \(I\) cannot be written as \(I = J_1 \cap J_2\), where \(J_1\) and \(J_2\) are monomial ideals with \(J_1 \supsetneq I\), \(J_2 \supsetneq I\) if and only if \(G(I)\) consists only of monomials of the form \(X_i^{\alpha_i}\) with \(\alpha_i > 0\) \footnote{Such monomials are called pure powers.}.
    \begin{proof}
    We will prove the direct implication using contraposition. Suppose that there exists at least one monomial in the generator set which is not a pure power, i.e. \(G(I) = \Set{ X_i^{\alpha_i} X_j^{\alpha_j}, m_1, \dots, m_n }\), with \(\alpha_i > 0\), \(\alpha_j > 0\), and \(m_1, \dots, m_n\) is a (possibly empty) set of monomials, such that \(m_k \not\mid X_i^{\alpha_i} X_j^{\alpha_j}\), \(\forall k \in \overline{1, n}\) (since then \(G(I)\) wouldn't be minimal). Let \(J_1 = (X_i^{\alpha_i}, m_1, \dots, m_n)\), \(J_2 = (X_j^{\alpha_j}, m_1, \dots, m_n)\). Then \(I = J_1 \cap J_2\) by the result from the previous exercise (note that \(X_i^{\alpha_i}\) and \(X_j^{\alpha_j}\) are coprime). In addition, \(I \subsetneq (X_i^{\alpha_i}, m_1, \dots, m_n)\), \(I \subsetneq (X_j^{\alpha_j}, m_1, \dots, m_n)\). This is the negation of the initial proposition.
    
    We will prove the converse by contradiction. Suppose \(I = J_1 \cap J_2\) and that \(I \subsetneq J_1\), \(I \subsetneq J_2\). From exercise 8 from the previous homework, we know that \(J_1 \cap J_2 = \left(\lcm(u, v) \mid u \in G(J_1), v \in G(J_2)\right)\). Since the inclusions are strict, we can take \(m_1 \in G(J_1) \setminus G(I)\), \(m_2 \in G(J_2) \setminus G(I)\). Then \(\lcm(m_1, m_2) \in J_1 \cap J_2 = I\), so there exists a pure power \(X_i^{\alpha_i} \in G(I)\) which divides \(\lcm(m_1, m_2)\). But then \(X_i^{\alpha_i} \divides m_1\) or \(X_i^{\alpha_i} \divides m_2\), contradiction.
    \end{proof}
    
    \item Decompose the ideal
    \[
        I = (X^3 Y, X^2 Y^4, Y^2 Z) \subset K[X, Y, Z]
    \]
    as an intersection of monomial ideals generated by pure powers.
    \begin{proof}
    We can use the first point of this problem to decompose a monomial ideal of the kind \((X_i^{\alpha_i} X_j^{\alpha_j}, m_1, \dots, m_n)\) into \((X_i^{\alpha_i}, m_1, \dots, m_n) \cap (X_j^{\alpha_j}, m_1, \dots, m_n)\).
    
    Therefore:
    \begin{align*}
        (X^3 Y, X^2 Y^4, Y^2 Z) &= (X^3, X^2 Y^4, Y^2 Z) \cap (Y, X^2 Y^4, Y^2 Z) \\
        &= (X^3, X^2, Y^2 Z) \cap (X^3, Y^4, Y^2 Z) \cap (Y, X^2, Y^2 Z) \cap (Y, Y^4, Y^2 Z) \\
        &= (X^3, X^2, Y^2) \cap (X^3, X^2, Z) \\
        &\quad \cap (X^3, Y^4, Y^2) \cap (X^3, Y^4, Z) \\
        &\quad \cap (Y, X^2, Y^2) \cap (Y, X^2, Z) \\
        &\quad \cap (Y, Y^4, Y^2) \cap (Y, Y^4, Z) \\
        &= (X^2, Y^2) \cap (X^2, Z) \cap (X^3, Y^2) \cap (X^3, Y^4, Z) \\
        &\quad \cap (X^2, Y) \cap (X^2, Y, Z) \cap (Y) \cap (Y, Z)
    \end{align*}
    
    Note that some of the primary ideals contain the intersection of the others. They can be removed without changing the value of the intersection. The end result is:
    \[
        (X^3 Y, X^2 Y^4, Y^2 Z) = (X^2, Z) \cap (X^3, Y^2) \cap (X^3, Y^4, Z) \cap (Y)
    \]
    \end{proof}
\end{enumerate}
\end{problem*}
