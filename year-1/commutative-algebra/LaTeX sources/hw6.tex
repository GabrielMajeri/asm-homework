\section*{Problem set 6}
\stepcounter{section}

\begin{comment}
\begin{problem*}{2}
~
\begin{proof}
We will label the three statements with \(a\), \(b\) and \(c\), and show that they are all equivalent by proving the following implications:
\begin{itemize}
    \item[\(a \implies b\)] Based on problem 7 from problem set 3, we know that we can decompose any monomial ideal \(I \leq K[X_1, \dots, X_n]\) as a finite intersection of monomial ideals generated by pure powers. We will write this decomposition as
    \[
        I = f_1 \cap f_2 \cap \dots \cap f_k
    \]
    By the characterization given in problem 7 from problem set 4, we know that each \(f_i\) is a primary ideal.
    
    We can remove superfluous ideals from the intersection and group together ideals with the same radical in order to get a minimal primary decomposition:
    \[
        I = q_1 \cap q_2 \cap \dots \cap q_{k'}
    \]
    where \(k' \leq k\). Note that while \(q_i\) might not be a monomial ideal generated by pure powers anymore, it is still a primary ideal generated by monomials. This shows that any monomial ideal has a minimal primary decomposition made out of primary monomial ideals.
    
    Now we use a result from the course which tells us that
    \[
        \dim R/I = \dim R/\sqrt{I}
    \]
    From this we get \(\dim R/\sqrt{I} = 0\).
    
    Using another result we have that
    \[
        \dim R/\sqrt{I} = \sup \Set{ \dim R/p_1, \dots, \dim R/p_{k'} }
    \]
    where we denote by \(p_i\) the corresponding associated prime \(p_i = \sqrt{q_i}\). The only possibility is \(\dim R/p_i = 0\) for all \(p_i\).
    
    Since \(K[X_1, \dots, X_n]\) is an affine \(K\)-domain, we have
    \[
        \height p_i + \dim R/p_i = \dim R
    \]
    for all \(p_i\). Therefore \(\height p_i = \dim R\), but by the definition of Krull dimension, this implies that each \(p_i\) is a maximal ideal.
    
    Furthermore, we know from the course that the radical of a monomial ideal is also a monomial ideal, so each \(p_i\) is a monomial ideal which is maximal in \(K[X_1, \dots, X_n]\).
    
    By problem 8 of problem set 1, the only maximal ideal generated by monomials is \((X_1, \dots, X_n)\). Therefore \(p_1 = p_2 = \dots = p_n = (X_1, \dots, X_n)\). But this means that
    \[
        \sqrt{I} = (X_1, \dots, X_n)
    \]
    which, by the maximality of \(X_1, \dots, X_n\), shows that \(I\) is \((X_1, \dots, X_n)\)-primary.
    
    \item[\(b \implies c\)] % TODO
    
    \item[\(c \implies a\)] % TODO
\end{itemize}
\end{proof}
\end{problem*}
\end{comment}

\begin{problem*}{3}
In the polynomial ring \(R = K[X, Y, Z]\) we consider the ideal
\[
    I = (X^3 Z^3, X^2 Y)
\]
\begin{enumerate}[(a)]
    \item Find \(\dim R/I\) and \(\height I\).
    \begin{proof}
    A statement from the course tells us that \(\dim R/I = \dim R/\sqrt{I}\). Since \(I\) is a monomial ideal, we can use another proposition which says that
    \[
        \sqrt{(m_1, \dots, m_k)} = \left(\sqrt{m_1}, \dots, \sqrt{m_k}\right)
    \]
    where \(m_1, \dots, m_k\) are monomials. Therefore
    \[
        \sqrt{I} = (XZ, XY)
    \]
    Using problem 7 from problem set 3, we can further decompose \(\sqrt{I}\) as
    \[
        \sqrt{I} = (X, XY) \cap (Z, XY) = (X) \cap (X, Y) \cap (Z, X) \cap (Z, Y)
    \]
    Removing the ideals which are not necessary in this intersection we are left with
    \[
        \sqrt{I} = (X) \cap (Z, Y)
    \]
    and since \(\sqrt{(X)} \neq \sqrt{(Z, Y)}\), this decomposition is minimal.
    
    Another result from the course allows us to compute the dimension of \(R/\sqrt{I}\) as follows:
    \begin{align*}
        \dim R/\sqrt{I} &= \max \Set{ \dim R/(X), \dim R/(Z, Y) } \\
        &= \max \Set{ \dim K[Y, Z], \dim K[X] } \\
        &= \max \Set{ 2, 1 } \\
        &= 2
    \end{align*}
    hence \(\dim R/I = 2\).
    
    Since \(K[X, Y, Z]\) is an affine \(K\)-domain, we get that
    \begin{align*}
        \height I &= \dim R - \dim R/I \\
        &= 3 - 2 \\
        &= 1
    \end{align*}
    \end{proof}
    
    \item Find \(\dim R/I^2\) and \(\height I^2\).
    \begin{proof}
    We have \(I^2 = I \cdot I\), therefore \(\sqrt{I^2} = \sqrt{I \cdot I} = \sqrt{I \cap I} = \sqrt{I}\). This shows that \(\dim R/I^2 = \dim R/\sqrt{I^2} = \dim R/\sqrt{I}\), which we've shown previously to be \(2\). By the same reasoning as above, \(\height{I^2} = 1\).
    \end{proof}
    
    \item Find a chain in \(\Spec(R/I)\) of length equal to \(\dim R/I\).
    \begin{proof}
    The prime ideals of the quotient ring \(R/I\) correspond to prime ideals of \(R\) containing \(I\). We remark that \((X)\), \((X, Y)\) and \((X, Y, Z)\) are all prime ideals in \(K[X, Y, Z]\) containing \(I\). Therefore we have a chain of length 2 in \(R/I\) composed of:
    \[
        (X) + I \subsetneq (X, Y) + I \subsetneq (X, Y, Z) + I
    \]
    \end{proof}
\end{enumerate}
\end{problem*}

\begin{problem*}{4}
Let \(R\) be a ring and \(a \in R\) which is not nilpotent. Show that \(R_a \cong R[X]/(aX - 1)\) (as rings).
\begin{proof}
Using the universal property of the polynomial ring, we define the \(R\)-algebra homomorphism \(\varphi \colon R[X] \to R_a\) by \(\varphi(X) = \frac{1}{a}\). We now prove that it's injective and surjective:
\begin{itemize}
    \item We claim that \(\ker \varphi = (aX - 1)\). We will prove this by double inclusion.
    
    \begin{itemize}
        \item[\(\supseteq\)] If \(f \in (aX - 1)\), then \(f = (aX - 1) g\) for some \(g \in R[X]\). But \(\varphi(aX - 1) = a \frac{1}{a} - 1 = 0\), which shows that \(\varphi(f) = 0\).
        
        \item[\(\subseteq\)] Let \(f \in \ker \varphi\). Since \(\varphi(f) = 0\), we get that \(f\left(\frac{1}{a}\right) = 0\), if we interpret \(f\) as a polynomial in \(R_a[X]\). Furthermore, the leading coefficient of \(aX - 1\) is an invertible element in \(R_a\), therefore we can apply the first step of the Euclidean division algorithm to \(f\) and \(aX - 1\) to obtain
        \[
            f = g (aX - 1) = (a X) g - g
        \]
        for some \(g \in R_a[X]\). But since \(f\) is also in \(R[X]\), \(g\) must also be in \(R[X]\) (if \(g\) had any fractional coefficient, then clearly \(f\) would also have one, since it couldn't be canceled by \((aX) g\)).
        
        Thus, \(f \in (aX - 1) R[X]\), as claimed.
    \end{itemize}

    \item Let \(y \in R_a\). Then \(y = \frac{r}{a^k}\) for some \(r \in R\), \(k \in \naturals\). Clearly \(r X^k \in R[X]\) and we have that
    \[
        \varphi\left(r X^k\right) = r \varphi\left(X^k\right) = r \varphi(X)^k = r \left(\frac{1}{a}\right)^k = r \, \frac{1}{a^k} = \frac{r}{a^k}
    \]
    Therefore \(\varphi\) is surjective.
\end{itemize}

\(\varphi\) is a bijection, hence \(R_a\) and \(R[X]/\ker \varphi = R[X]/(aX - 1)\) are isomorphic as \(R\)-algebras, and therefore also as rings.
\end{proof}
\end{problem*}

\begin{problem*}{6}
~
\begin{enumerate}[(a)]
    \item Let \(M, N\) be \(R\)-modules and \(f \colon M \to N\) be an \(R\)-linear map. If \(S\) is any multiplicative set in \(R\), then the map
    \begin{gather*}
        S^{-1} f \colon S^{-1} M \to S^{-1} N \\
        \frac{m}{s} \mapsto \frac{f(m)}{s}
    \end{gather*}
    is well-defined and it is a homomorphism of \(S^{-1} R\)-modules.
    
    \begin{proof}
    First, we shall check that \(S^{-1} f\) is well-defined. Let \(\frac{m}{s} = \frac{m'}{s'}\). We want to show that \(\frac{f(m)}{s} = \frac{f(m')}{s'}\).
    
    By definition, \((m, s) \sim (m', s')\) iff \(\exists u \in S\) such that \(u (s'm - sm') = 0\). Applying \(f\) to this expression and using the fact that it's \(R\)-linear, we obtain:
    \begin{align*}
        f(u(s'm - sm')) &= f(0) \iff \\
        u f(s'm - sm') &= 0 \iff \\
        u (f(s'm) - f(sm')) &= 0 \iff \\
        u (s' f(m) - s f(m')) &= 0
    \end{align*}
    which shows that \((f(m), s) \sim (f(m'), s')\), as required.
    
    Now we will prove that it's an \(S^{-1}R\)-module homomorphism.
    \begin{itemize}
        \item Let \(\frac{a}{s}, \frac{b}{t} \in S^{-1} M\). Then
        \begin{gather*}
            S^{-1}f\left(\frac{a}{s} + \frac{b}{t}\right)
            = S^{-1} f\left(\frac{t a + s b}{st}\right)
            = \frac{f(ta + sb)}{st}
            = \frac{tf(a) + sf(b)}{st} = \\
            \smallskip
            = \frac{f(a)}{s} + \frac{f(b)}{t}
            = S^{-1} f \left(\frac{a}{s}\right) + S^{-1} f \left(\frac{b}{t}\right)
        \end{gather*}
        
        \item Let \(r \in R\), \(\frac{a}{s} \in S^{-1} M\). Then
        \begin{gather*}
            S^{-1} f \left(r \, \frac{a}{s}\right) = S^{-1} f \left(\frac{ra}{s}\right) = \frac{f(ra)}{s} = \frac{r f(a)}{s} = r \, \frac{f(a)}{s} = r \, S^{-1} f \left(\frac{a}{s}\right)
        \end{gather*}
    \end{itemize}
    \end{proof}
    
    \item Assume that the \(R\)-linear maps \(f \colon M_1 \to M_2\) and \(g \colon M_2 \to M_3\) satisfy \(\Ima f \subseteq \ker g\). Prove that the induced maps \(S^{-1} f \colon S^{-1} M_1 \to S^{-1} M_2\) and \(S^{-1} g \colon S^{-1} M_2 \to S^{-1} M_3\) satisfy \(\Ima (S^{-1} f) \subseteq \ker (S^{-1} g)\). Moreover, show that if \(\Ima f = \ker g\), then \(\Ima (S^{-1} f) = \ker (S^{-1} g)\).
    
    \begin{proof}
    Let \(\frac{m}{s} \in \Ima S^{-1} f\). We want to show that \(S^{-1} g\left(\frac{m}{s}\right) = 0\). By the definition of the image, \(\exists \frac{m'}{s'} \in S^{-1} M_1\) such that \(S^{-1} f \left(\frac{m'}{s'}\right) = \frac{m}{s}\). This means that \((f(m'), s') \sim (m, s)\), hence \(\exists u \in S\) such that
    \[
        u(s f(m') - s' m) = 0
    \]
    Applying \(g\) to the equation above and using its linearity, we get
    \begin{align*}
        g(u(s f(m') - s' m)) &= g(0) \iff \\
        u(g(s f(m') - s' m)) &= 0 \iff \\
        u(g(s f(m')) - g(s' m)) &= 0 \iff \\
        u(s g (f(m')) - s' g (m)) &= 0 \iff \\
        u(s \cdot 0 - s' g (m)) &= 0
    \end{align*}
    where on the last line we've used the fact that \(\Ima f \subseteq \ker g\), which we know from the hypothesis. This shows that \(\frac{g(m)}{s} = 0 \iff S^{-1} g \left(\frac{m}{s}\right) = 0\), hence \(\frac{m}{s} \in \ker S^{-1} g\) as desired.
    
    If we strengthen the hypothesis to \(\Ima f = \ker g\), we also get the reverse inclusion. Let \(\frac{m}{s} \in \ker S^{-1} g\). We need to show that \(\frac{m}{s}\) is the image of something from \(S^{-1} M_1\) through \(f\). By the definition of the kernel of a homomorphism, we get that \(S^{-1} g\left(\frac{m}{s}\right) = 0\). By the definition of \(S^{-1} g\), we deduce that \(\frac{g(m)}{s} = 0\). Hence \((g(m), s) \sim (0,1)\), which means \(\exists u \in S\) such that
    \[
        u(1 \cdot g(m) - s \cdot 0) = 0 \iff
        u \cdot g(m) = 0 \iff
        g(um) = 0
    \]
    from which we deduce that \(um \in \ker g\).
    
    Now, since \(\ker g = \Ima f\), we know that \(um \in \Ima f\). Take \(m' \in M_1\) such that \(um = f(m')\). But then \(\frac{um'}{us} = \frac{m'}{s} \in S^{-1} M_1\) and
    \[
        S^{-1} f \left(u \cdot \frac{m'}{s}\right) = u \cdot S^{-1} f \left(\frac{um'}{us}\right) = u \cdot \frac{f(um')}{us} = u \cdot \frac{m}{us} = \frac{m}{s}
    \]
    which shows that \(\ker S^{-1} g \subseteq \Ima S^{-1} f\).
    \end{proof}
\end{enumerate}
\end{problem*}
