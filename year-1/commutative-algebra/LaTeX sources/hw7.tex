\section*{Course 8 homework}
\stepcounter{section}

\begin{exercise}
Let \(R\) be a ring and \(S \subseteq R\) a multiplicatively closed set. Denote by \(\varepsilon\) the canonical homomorphism from \(R\) to \(S^{-1} R\).

\begin{enumerate}
    \item Every ideal in \(S^{-1} R\) is an extended ideal. That is, every ideal is of the form \(S^{-1} I\) with \(I \leq R\), where \(S^{-1} I = \Set{ \frac{a}{s} \vert a \in I, s \in S }\).
    
    \begin{proof}
    Let \(J\) be an ideal of \(S^{-1} R\). Denote by \(N_J\) the set of all the possible numerators of \(J\), i.e. \(N_J = \Set{ a \in R \mid \exists s \in S \text{ s.t. } \frac{a}{s} \in J}\). We are going to prove that \(N_J\) is an ideal of \(R\).
    
    First, notice that if \(\frac{a}{s} \in J\), then
    \[
        \frac{s}{1} \cdot \frac{a}{s} = \frac{a}{1} \in J
    \]
    since \(J \leq S^{-1} R\). Hence, if \(a \in N_J\), then certainly \(\frac{a}{1} \in J\).
    
    Let \(a, b \in N_J\). Then
    \[
        \frac{a}{1} - \frac{b}{1} = \frac{a - b}{1} \in J
    \]
    which shows that \(a - b \in N_J\).
    
    Let \(r \in R\) and \(a \in N_J\). Then
    \[
        \varepsilon(r) \frac{a}{1} = \frac{r}{1} \cdot \frac{a}{1} = \frac{ra}{1} \in J
    \]
    which shows that \(ra \in N_J\).
    
    Hence, \(N_J \leq R\) and, by the way it has been defined, \(S^{-1} N_J = J\).
    \end{proof}
    
    \item If \(\underline{a} \leq R\), then
    \[
        \varepsilon^{-1} (\underline{a} \, S^{-1} R) = \bigcup_{s \in S} \, \underline{a} : (s)
    \]
    Thus \(\underline{a} \, S^{-1} R = S^{-1} R \iff \underline{a} \cap S \neq \emptyset\).
    
    \begin{proof}
    We will prove this equality by double inclusion.
    
    \begin{itemize}
        \item[\(\subseteq\)] Let \(x \in \varepsilon^{-1} (\underline{a} \, S^{-1} R)\).
        This means that
        \[
            \varepsilon(x) \in \underline{a} \, S^{-1} R \iff
            \frac{x}{1} = \frac{a}{s}
        \]
        for some \(a \in \underline{a}\) and \(s \in S\). As we've seen in the previous exercise, because \(\underline{a} \, S^{-1} R\) is an ideal, we can strengthen this to
        \[
            \frac{x}{1} = \frac{a}{1}
        \]
        for some (possibly other) element \(a \in \underline{a}\).
        
        By the definition of equality in the localization, it means that \(\exists s \in S\) such that
        \[
            s(1x - 1a) = 0 \iff s x = s a
        \]
        But this simply shows that
        \[
            x \cdot (s) \subseteq (a) \subseteq \underline{a}
        \]
        hence \(x \in \underline{a} : (s)\) for some \(s \in S\).
        
        \item[\(\supseteq\)] Let \(y \in \underline{a} : (s)\) for some \(s \in S\). We want to show that \(y \in \varepsilon^{-1}(\underline{a} \, S^{-1} R)\). It's enough to check that
        \[
            \varepsilon(y) = \frac{y}{1} \in \underline{a} \, S^{-1} R
        \]
        
        By the definition of the colon ideal, \(y \cdot (s) \subseteq \underline{a}\). In particular, \(y s \in \underline{a}\). Hence
        \[
            \frac{y}{1} = \frac{ys}{s} \in \underline{a} \, S^{-1} R
        \]
        as desired.
    \end{itemize}
    \end{proof}
    
    \item The prime ideals of \(S^{-1} R\) are in one-to-one correspondence with the prime ideals of \(R\) which do not intersect \(S\).
    
    \begin{proof}
    Let \(q\) be an arbitrary prime ideal of \(S^{-1} R\).
    
    We know that \(\varepsilon^{-1}(q)\) is a prime ideal in \(R\) (the preimage of a prime ideal through a ring homomorphism is prime).
    
    It does not intersect \(S\), since otherwise we could take \(s \in S \cap \varepsilon^{-1}(q)\) and then \(\varepsilon(s) \in q\) would be an invertible element, contradicting the primality of \(q\). 
    
    This shows that the mapping we've built so far is surjective. To show that it is also injective, consider two arbitrary prime ideals \(q, q' \in \Spec(S^{-1} R)\) such that \(q \neq q'\). This means that \(\exists \, \frac{a}{s}\) with \(\frac{a}{s} \in q\) but \(\frac{a}{s} \not\in q'\). Hence by the correspondence between the ideals of \(R\) and the ideals of \(S^{-1} R\), we get that \(\varepsilon^{-1}(q) \neq \varepsilon^{-1}(q')\).
    \end{proof}
    
    \item \(S^{-1}\) commutes with finite sums, products, intersections and radicals.
    
    \begin{proof}
    It's enough to show this happens for two arbitrary ideals \(I\) and \(J\), then the result can be extended to the finite case by induction.
    
    We will prove that \(S^{-1} (I + J) = S^{-1} I + S^{-1} J\) using double inclusion.
    \begin{itemize}
        \item[\(\subseteq\)] Let \(\frac{a + b}{s} \in S^{-1}(I + J)\). Then
        \[
            \frac{a + b}{s} = \frac{a}{s} + \frac{b}{s} \in S^{-1} I + S^{-1} J
        \]
        
        \item[\(\supseteq\)] Let \(\frac{a}{s} \in S^{-1} I\), \(\frac{b}{u} \in S^{-1} J\). Then
        \[
            \frac{a}{s} + \frac{b}{u} = \frac{u a + s b}{s u} \in S^{-1} (I + J)
        \]
        where \(ua \in I\), \(sb \in J\) since \(I, J\) are ideals.
    \end{itemize}
    
    We will do the same for \(S^{-1} (I \cdot J) = S^{-1} I \cdot S^{-1} J\).
    \begin{itemize}
        \item[\(\subseteq\)] Let \(\frac{ab}{s} \in S^{-1} (I \cdot J)\). Then
        \[
            \frac{ab}{s} = \frac{a}{s} \cdot \frac{b}{1} \in S^{-1} I \cdot S^{-1} J
        \]
        
        \item[\(\supseteq\)] Let \(\frac{a}{s} \in S^{-1} I\), \(\frac{b}{u} \in S^{-1} J\). Then
        \[
            \frac{a}{s} \cdot \frac{b}{u} = \frac{ab}{su} \in S^{-1} (I \cdot J)
        \]
    \end{itemize}
    
    And again for \(S^{-1} (I \cap J) = S^{-1} I \cap S^{-1} J\).
    \begin{itemize}
        \item[\(\subseteq\)] Let \(\frac{a}{s} \in S^{-1} (I \cap J)\). Then \(a \in I \cap J\), hence \(a \in I\) and \(a \in J\), which means that \(\frac{a}{s} \in S^{-1} I\) and \(\frac{a}{s} \in S^{-1} J\).
        
        \item[\(\supseteq\)] Let \(\frac{a}{s} \in S^{-1} I\) and \(\frac{b}{t} \in S^{-1} J\) with \(\frac{a}{s} = \frac{b}{t}\). Then \(\exists u \in S\) such that \(u(ta - sb) = 0 \iff uta = usb\), with \(a \in I\) and \(b \in J\). Hence \(uta \in I \cap J\) and \(usb \in I \cap J\), so \(\frac{a}{s} = \frac{uta}{uts} \in S^{-1} (I \cap J)\).
    \end{itemize}
    
    For an arbitrary \(I \leq R\), we want \(S^{-1} \sqrt{I} = \sqrt{S^{-1} I}\).
    \begin{itemize}
        \item[\(\subseteq\)] Let \(\frac{a}{s} \in S^{-1} \sqrt{I}\). This means that \(a \in \sqrt{I}\), therefore \(\exists n \in \naturals\) such that \(a^n \in I\). Then \(\frac{a^n}{1} \in S^{-1} I \subseteq \sqrt{S^{-1} I}\).
    
        \item[\(\supseteq\)] Let \(\frac{a}{s} \in \sqrt{S^{-1} I}\). This means that \(\exists n \in \naturals\) such that \(\left(\frac{a}{s}\right)^n \in S^{-1} I\). Hence \(\frac{a^n}{s^n} \in S^{-1} I\), from which we deduce that \(a^n \in I\). Therefore \(a \in \sqrt{I}\).
    \end{itemize}
    \end{proof}
    
    \item If
    \[
        \begin{tikzcd}
            0 \arrow[r] & M_1 \arrow[r, "f"] & M_2 \arrow[r, "g"] & M_3 \arrow[r] & 0
        \end{tikzcd}
    \]
    is a short exact sequence of \(R\)-modules, then
    \[
        \begin{tikzcd}
            0 \arrow[r] & S^{-1} M_1 \arrow[r, "S^{-1} f"] & M_2 \arrow[r, "S^{-1} g"] & M_3 \arrow[r] & 0
        \end{tikzcd}
    \]
    is a short exact sequence of \(S^{-1} R\)-modules.
    
    \begin{proof}
    We are going to prove that \(\Ima S^{-1} f = \ker S^{-1} g\) using double inclusion.
    
    \begin{itemize}
        \item[\(\subseteq\)] Let \(\frac{m}{s} \in \Ima S^{-1} f\). We want to show that \(S^{-1} g\left(\frac{m}{s}\right) = 0\). By the definition of the image, \(\exists \frac{m'}{s'} \in S^{-1} M_1\) such that \(S^{-1} f \left(\frac{m'}{s'}\right) = \frac{m}{s}\). This means that \((f(m'), s') \sim (m, s)\), hence \(\exists u \in S\) such that
        \[
            u(s f(m') - s' m) = 0
        \]
        Applying \(g\) to the equation above and using its linearity, we get
        \begin{align*}
            g(u(s f(m') - s' m)) &= g(0) \iff \\
            u(g(s f(m') - s' m)) &= 0 \iff \\
            u(g(s f(m')) - g(s' m)) &= 0 \iff \\
            u(s g (f(m')) - s' g (m)) &= 0 \iff \\
            u(s \cdot 0 - s' g (m)) &= 0
        \end{align*}
        where on the last line we've used the fact that \(\Ima f \subseteq \ker g\), which we know to be true since the original sequence is exact. This shows that
        \[
            \frac{g(m)}{s} = 0 \implies S^{-1} g \left(\frac{m}{s}\right) = 0
        \]
        hence \(\frac{m}{s} \in \ker S^{-1} g\) as desired.
        
        \item[\(\supseteq\)] Let \(\frac{m}{s} \in \ker S^{-1} g\). We need to show that \(\frac{m}{s}\) is the image of something from \(S^{-1} M_1\) through \(f\). By the definition of the kernel of a homomorphism, we get that \(S^{-1} g\left(\frac{m}{s}\right) = 0\). By the definition of \(S^{-1} g\), we deduce that \(\frac{g(m)}{s} = 0\). Hence \((g(m), s) \sim (0,1)\), which means \(\exists u \in S\) such that
        \[
            u(1 \cdot g(m) - s \cdot 0) = 0 \iff
            u \cdot g(m) = 0 \iff
            g(um) = 0
        \]
        from which we deduce that \(um \in \ker g\).
        
        Now, since \(\ker g = \Ima f\), we know that \(um \in \Ima f\). Take \(m' \in M_1\) such that \(um = f(m')\). But then \(\frac{um'}{us} = \frac{m'}{s} \in S^{-1} M_1\) and
        \[
            S^{-1} f \left(u \cdot \frac{m'}{s}\right) = u \cdot S^{-1} f \left(\frac{um'}{us}\right) = u \cdot \frac{f(um')}{us} = u \cdot \frac{m}{us} = \frac{m}{s}
        \]
        which shows that \(\ker S^{-1} g \subseteq \Ima S^{-1} f\).
    \end{itemize}
    \end{proof}
    
    \begin{comment}
    \item If \(I, J \leq R\) and \(J\) is finitely generated, then \(S^{-1} (I : J) = S^{-1} I : S^{-1} J\). Give a counterexample to show that \(J\) finitely generated is essential.
    
    \begin{proof}
    We will prove this using double inclusion.
    \begin{itemize}
        \item[\(\subseteq\)] Let \(\frac{a}{s} \in S^{-1} (I : J)\). Then \(a \in I : J\).
        
        Take an arbitrary \(\frac{b}{t} \in S^{-1} J\). We have
        \[
            \frac{a}{1} \cdot \frac{b}{t} = \frac{\overbrace{a \cdot b}^{\in \, I}}{t} \in S^{-1} I
        \]
        hence \(\frac{a}{1} \in S^{-1} I : S^{-1} J\). But since the colon ideal of two ideals is also an ideal, we can multiply \(\frac{a}{1}\) by \(\frac{1}{s}\) to obtain
        \[
            \frac{1}{s} \cdot \frac{a}{1} = \frac{a}{s} \in S^{-1} I : S^{-1} J
        \]
        as desired.
        
        \item[\(\supseteq\)] Let \(\frac{a}{s} \in S^{-1} I : S^{-1} J\). To show that \(\frac{a}{s} \in S^{-1} (I : J)\), it's enough to check that \(a \in I : J\).
        
        By the definition of the colon ideal, we know that
        \[
            \frac{a}{s} \cdot S^{-1} J \subseteq S^{-1} I
        \]
        i.e. \(\forall b \in S^{-1} J\), \(\exists c \in S^{-1} I\) such that
        \[
            \frac{a}{s} \cdot \frac{b}{t} = \frac{c}{u}
        \]
    \end{itemize}
    \end{proof}
    \end{comment}
    
    \item[7.] If \(M\) is a finitely generated \(R\)-module, then \(S^{-1} (\Ann(M)) = \Ann(S^{-1} M)\). Give a counterexample to show that \(M\) finitely generated is necessary.
    
    \begin{proof}
    This proof is inspired by the one in Atiyah--MacDonald.
    
    Since \(M\) is finitely generated, we can express it as
    \[
        M = R a_1 + R a_2 + \dots + R a_n
    \]
    for some \(a_1, a_2, \dots, a_n \in R\). Thus, we can prove the statement by induction.
    
    Let \(M' = R a_1\). Consider the short exact sequence:
    \[
        \begin{tikzcd}
            0 \arrow[r] & \Ann(M') \arrow[hookrightarrow, r] & R \arrow[twoheadrightarrow, r, "\cdot \, a_1"] & M' \arrow[r] & 0
        \end{tikzcd}
    \]
    Since \(S^{-1}\) is an exact functor, the following sequence is also exact:
    \[
        \begin{tikzcd}
            0 \arrow[r] & S^{-1} \Ann(M') \arrow[hookrightarrow, r] & S^{-1} R \arrow[twoheadrightarrow, r] & S^{-1} M' \arrow[r] & 0
        \end{tikzcd}
    \]
    Thus \(S^{-1} M' \cong (S^{-1}R) / (S^{-1} \Ann (M'))\). Taking the annihilator of both sides tells us that \(\Ann (S^{-1} M') \cong S^{-1} (\Ann(M'))\).

    Now for the inductive step, let suppose we know the statement is true for all \(k < n\), and we want to prove it for \(k = n\). Then
    \begin{align*}
        S^{-1} (\Ann(R a_1 + \dots + R a_n)) &= S^{-1} (\Ann(R a_1) \cap \Ann(R a_2 + \dots + R a_n)) \\
        &= S^{-1} (\Ann(R a_1)) \cap S^{-1} (\Ann(R a_2 + \dots + R a_n)) \\
        &= \Ann (S^{-1} (R a_1)) \cap \Ann(S^{-1} (R a_2 \dots + R a_n)) \\
        &= \Ann (S^{-1} (R a_1) + S^{-1} (R a_2 + \dots + R a_n)) \\
        &= \Ann (S^{-1} (R a_1 + \dots + R a_n))
    \end{align*}
    where we've used the fact that the annihilator of a finite sum is the intersection of the annihilators, and that \(S^{-1}\) commutes with finite sums and finite intersections.
    
    Thus, for any finitely generated module \(M\), we get that \(S^{-1} (\Ann (M)) = \Ann(S^{-1} (M))\).
    
    For a counterexample, consider \(\integers\) as a \(\integers\)-module, with the quotient rings \(\integers_n, \forall n \in \naturals^*\) viewed as \(\integers\)-submodules. Define \(M\) to be
    \[
        M = \bigoplus_{n \in \naturals^*} \integers_n
    \]
    An element of \(M\) is a linear combination of finitely many non-zero terms from the various \(\integers_n\)s, with coefficients in \(\integers\).
    
    We remark that \(M\) is not finitely generated. Suppose \(a_1, \dots, a_k\) is a finite set of generators for \(M\). For each \(a_i\), denote by \(n_i\) the index of the largest component which is not \(0\) (\(n_i < \infty\) due to the properties of the direct sum). Let \(N = \max_{i = \overline{1, k}} \Set{ n_i }\). Then the element of \(M\) with \(1\) on its \(N\)-th position and zero everywhere else cannot be written as a linear combination of \(a_1, \dots, a_k\).
    
    If \(a m = 0\) for all \(m \in M\), it means that \(a \equiv 0 \mod k, \forall k \in \naturals^*\), hence \(a = 0\). In other words, \(\Ann(M) = 0\) and \(S^{-1} (\Ann(M)) = 0\).
    
    Let \(S = \integers^*\). Then \(S^{-1} M\) is an \(S^{-1} \integers \cong \rationals\)-module. We claim that \(S^{-1} M = 0\). For any \(\frac{m}{k} \in S^{-1} M\) which is not zero, take \(c \in \naturals^* \subseteq \integers^* = S\) to be the least common multiple of all of the indices of the non-zero components of \(m\). Then
    \[
        c (1 \cdot m - 0 \cdot k) = cm = 0
    \]
    since \(c \cdot a \equiv 0 \mod n\) if \(c\) is a multiple of \(n\). Thus, \(\Ann(S^{-1} M) = \Ann(0) = \rationals\).
    
    We conclude that
    \[
        0 = S^{-1} (\Ann(M)) \neq \Ann(S^{-1} M) = \rationals
    \]
    \end{proof}
    
    \item[8.] Prove that \(\dim (S^{-1} R) \leq \dim (R)\).
    
    \begin{proof}
    If \(\dim(R) = \infty\), then the inequality is true.
    
    Otherwise, as we've previously shown, the prime ideals of \(S^{-1} R\) are the prime ideals of \(R\) which do not intersect \(S\). Thus, there cannot be more prime ideals in \(S^{-1} R\) than in \(R\), and therefore no chain of primes longer than the longest one in \(R\).
    
    The equality case happens if there exists at least one chain of maximal length in \(R\) composed of primes which do not intersect \(S\).
    \end{proof}
\end{enumerate}

\end{exercise}

\begin{comment}
\begin{problem}
Let \(\psi \colon R_1 \to R_2\) be a surjective ring homomorphism and \(I \in \Spec(R_1)\) (or \(I \in \SpecMax(R_2)\)). Is \(\psi(I) \in \Spec(R_2)\)? (or \(\psi(I) \in \SpecMax(R_2)\) respectively?)

\begin{proof}
~
\begin{itemize}
    \item If \(\ker \psi \subseteq I\), then the statement is true.
    
    Let \(\alpha, \beta \in R_2\) such that \(\alpha \beta \in \psi(I)\). We can pick \(a, b \in R_1\) and \(c \in \psi(I)\) such that \(\alpha = \psi(a)\), \(\beta = \psi(b)\) and \(\psi(a) \psi(b) = \psi(c)\). Hence \(\psi(c) - \psi(a) \psi(b) = \psi(c) - \psi(ab) = \psi(c - ab) = 0\). Since \(I \supseteq \ker \psi\), there exists a \(d \in I\) such that \(c - ab = d\). Reordering, we get \(ab = c + d \in I\), and by the primality of \(I\), at least one of \(a\) or \(b\) must be in \(I\). Then at least one of \(\psi(a)\) or \(\psi(b)\) is in \(\psi(I)\).

    In this scenario, maximality is also preserved, since otherwise \(\psi(I) \subsetneq J \in \Spec(R_2)\) would imply \(I \subsetneq \psi^{-1}(J)\), where \(\psi^{-1}(J) \in \Spec(R_1)\) (by another result from the course).
    
    \item To show that the statement is false if \(\ker \psi \not\subseteq I\), consider the canonical projection \(p\) between \(\integers\) and \(\integers_2\). We have \(\ker p = 2\integers \not\subseteq 3\integers \in \Spec(\integers)\), yet \(p(3 \integers) = \integers_2\) (the whole ring), which clearly cannot be prime.
\end{itemize}
\end{proof}
\end{problem}
\end{comment}
