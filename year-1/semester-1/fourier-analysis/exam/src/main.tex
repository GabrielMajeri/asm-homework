\documentclass[a4paper, 12pt]{article}

% Better font specification support
\usepackage{fontspec}
% Encode mathematical symbols using UTF-8
\usepackage{unicode-math}
% Additional options for enumerations
\usepackage{enumerate}

% Useful math commands
\usepackage{amsmath}
% Additional options for defining custom theorem-like environments
\usepackage{amsthm}
% Mathematical tools
\usepackage{mathtools}

% Commands for absolute value and norm symbols
\DeclarePairedDelimiter\abs{\lvert}{\rvert}%
\DeclarePairedDelimiter\norm{\lVert}{\rVert}%

\newcommand*\diff{\mathop{}\!\symrm{d}}

\let\solution\proof
\let\endsolution\endproof

\renewcommand*{\proofname}{Solution}

\newcommand*{\reals}{\symbb{R}}
\newcommand*{\complex}{\symbb{C}}

\DeclareMathOperator{\supp}{supp}

\title{Elements of Analysis and Fourier Analysis \\[0.5em] Exam}
\author{Gabriel Majeri}
\date{30.01.2022}

\begin{document}

\maketitle

\section*{Exercise I}

Let \(f \colon \reals \to \reals\) be the \(2 \pi\)-periodic function on \(\reals\) such that
\[
    f(x) = \begin{cases}
        x, \quad - \pi \leq x < 0 \\[0.5em]
        0, \quad 0 \leq x \leq \pi
    \end{cases}
\]
Find the Fourier series of \(f\). What can you say about the pointwise, uniform and \(L^2\) convergence of the sequence \(\left(S_n(f)\right)_{n \, \geq \, 1}\) of partial sums of the Fourier series?

\begin{proof}
The Fourier series of \(f\) can be computed as
\begin{align*}
    \widehat{f}(n) &= \frac{1}{2\pi} \int_{-\pi}^{+\pi} f(x) e^{-i n x} \diff x \\[0.5em]
    &= \frac{1}{2\pi} \left(\int_{-\pi}^0 x e^{-inx} \diff x + \int_{0}^{\pi} 0 e^{-inx} \diff x \right) \\[0.5em]
    &= \frac{1}{2\pi} \int_{-\pi}^{0} x e^{-inx} \diff x \\[0.5em]
    &= \frac{1}{2\pi} \frac{1}{-in} \int_{-\pi}^{0} x (e^{-inx})' \diff x \\[0.5em]
    &= - \frac{1}{2\pi i n} \left(\left.x e^{-inx}\right|_{-\pi}^{0} - \int_{-\pi}^{0} e^{-inx} \diff x\right) \\[0.5em]
    &= - \frac{1}{2\pi i n} \left(\pi e^{i n \pi} - \left.\frac{i e^{-i n x}}{n}\right|_{-\pi}^{0} \right) \\[0.5em]
    &= - \frac{1}{2 \pi i n} \left(\pi e^{i n \pi} - \left(\frac{i}{n} - \frac{i e^{in \pi}}{n}\right)\right) \\[0.5em]
    &= -\frac{\pi e^{in\pi}}{2\pi i n} + \frac{i - i e^{i n \pi}}{2\pi i n^2} \\[0.5em]
    &= \frac{i - i e^{in\pi} - n \pi e^{in\pi}}{2\pi i n^2} \\[0.5em]
    &= \frac{1 - e^{in\pi} - n \pi e^{in\pi - i \frac{\pi}{2}}}{2 \pi n^2}
\end{align*}
for \(n \neq 0\), and it is
\[
    \widehat{f}(0) = \frac{1}{2\pi} \int_{-\pi}^{+\pi} f(x) \diff x = -\frac{\pi^2}{2}
\]
for \(n = 0\).

The partial sums of the Fourier series are
\[
    S_N (f, x) = \sum_{n = -N}^{N} \widehat{f}(n) e^{i n x}
\]

We will show that the partial sums of the Fourier series converge pointwise everywhere inside the domain \((-\pi, \pi)\), and to do so we will use Dirichlet's theorem.

Note that \(f \in L^1 [-\pi, \pi]\), since
\[
    \int_{-\pi}^{\pi} \abs{f(x)} \diff x = \int_{-\pi}^{0} \abs{x} \diff x = \frac{\pi^2}{2}
\]

Furthermore, \(f\) is continuous inside \((-\pi, \pi)\), hence we have
\[
    \lim_{N \to \infty} S_N (f, x) = f(x), \forall x \in (-\pi, \pi)
\]

At the margins of the interval, the lateral limits \(f(\pm \pi + 0)\), \(f(\pm \pi - 0)\) exist, and the lateral derivatives to so as well, hence again by Dirichlet we have
\begin{align*}
    \lim_{N \to \infty} S_N (f, \pm \pi) &= \frac{1}{2} (f(\pi + 0) + f(\pi - 0)) \\
    &= \frac{1}{2} (0 - \pi) = - \frac{\pi}{2}
\end{align*}

Using a theorem from course 7, we also have that for a piecewise \(C^1\) function \(f\) on \(L^1 [-\pi, \pi]\), the partial sums converge uniformly. We have lateral limits of \(f\) and of \(f'\) everywhere, and hence the uniform convergence works on the whole domain.
\end{proof}

\section*{Exercise II}

Let \(f, g \colon \reals \to \reals\) be two functions in \(L^1 (\reals) \cap L^2 (\reals)\). Prove that
\[
    \int_{\reals} f(x) g(x) \diff x = \frac{1}{2 \pi} \int_{\reals} \hat{f}(t) \hat{g}{t} \diff t
\]

\begin{proof}
\(L^2 (\reals)\) is an inner product space. Hence, if we want to compute the square of the inner product of \(f\) and \(g\), it is equal to
\[
    (f \mid \overline{g})^2 = \int_{\reals} f(x) g(x) \diff x
\]

Furthermore, since \(f\) and \(g\) are both in \(L^1(\reals) \cap L^2(\reals)\), if we use the Hilbert space isomorphism \(T\) from Plancherel's theorem (which respects the inner norm), we get that the result above is also equal to
\[
    (f \mid g)^2 = (\mathop{T} f \mid \mathop{T} \overline{g})^2 = \int_{\reals} \mathop{T} f(x) \mathop{T} g(x) \diff x
\]
but, using the definition of \(T\), this is simply
\[
    = \int_{\reals} \frac{1}{\sqrt{2 \pi}} \widehat{f}(t) \frac{1}{\sqrt{2 \pi}} \widehat{g}(t) \diff t = \frac{1}{2\pi} \int_{\reals} \widehat{f}(t) \widehat{g}(t) \diff t
\]
as desired.
\end{proof}

\section*{Exercise III}

Use the Fourier series to compute the integral
\[
    \int_{- \infty}^{+ \infty} \left(\frac{\sin (2x)}{x}\right)^3 \diff x
\]

\begin{proof}
Let \(\varphi_{a} = \chi_{[-a, a]}\), as defined in course 12. Then, using the result from the course, we have that
\[
    \widehat{\varphi_{a}} (t) = \frac{2 \sin (ta)}{t}
\]
In addition, from the same course we have that
\[
    k_a = \varphi_{a} * \varphi_{a} (x) = \begin{cases}
        2a + x, x \in [-2a, 0] \\
        2a - x, x \in [0, 2a] \\
        0, \abs{x} > 2a
    \end{cases}
\]

We have that
\begin{align*}
    \int_{-\infty}^{+\infty} \left(\frac{\sin (2x)}{x}\right)^3 \diff x &= \int_{-\infty}^{+\infty} \frac{\sin (2x)}{x} \cdot \frac{\sin (2x)}{x} \cdot \frac{\sin (2x)}{x} \diff x \\
    &= \frac{1}{8} \int_{-\infty}^{+\infty} \frac{2 \sin (2x)}{x} \cdot \frac{2 \sin (2x)}{x} \cdot \frac{2 \sin (2x)}{x} \diff x \\
    &= \frac{1}{8} \int_{-\infty}^{+\infty} \widehat{\varphi_2}(t) \cdot \widehat{\varphi_2}(t) \cdot \widehat{\varphi_2}(t) \diff t \\
    &= \frac{1}{8} \int_{-\infty}^{+\infty} (\widehat{\varphi_2 (t) * \varphi_2 (t)}) \cdot \widehat{\varphi_2} (t) \diff t \\
    &= \frac{1}{8} \int_{-\infty}^{+\infty} \widehat{k_2} (t) \cdot \widehat{\varphi_2} (t) \diff t
\end{align*}

Since we clearly have \(k_2\) and \(\varphi_2 \in L^1 (\reals) \cap L^2 (\reals)\), using the result from the previous exercise we have that
\begin{align*}
    \frac{1}{8} \int_{-\infty}^{+\infty} \widehat{k_2} (t) \cdot \widehat{\varphi_2} (t) \diff t &= \frac{1}{8} \int_{-\infty}^{+\infty} k_2 (x) \cdot \varphi_2 (x) \diff x
    \\
    &= \frac{1}{8} \frac{1}{2\pi} \int_{-\infty}^{+\infty} k_2 (x) \cdot \varphi_2 (x) \diff x \\
    &= \frac{1}{16 \pi} \int_{-2}^{2} k_2 (x) \diff x \\
    &= \frac{1}{16 \pi} \left(\int_{-2}^{0} k_2 (x) \diff x + \int_{0}^{2} k_2(x) \diff x\right) \\
    &= \frac{1}{16 \pi} \left(\int_{-2}^{0} (4 + x) \diff x + \int_{0}^{2} (4 - x) \diff x\right) \\
    &= \frac{1}{16\pi} \left(\left.\left(4x + \frac{x^2}{2}\right)\right|_{-2}^{0} + \left.\left(4x - \frac{x^2}{2}\right)\right|_{0}^{2}\right) \\
    &= \frac{1}{16\pi} ((0 + 0) - (-8 + \frac{4}{2}) + (8 - \frac{4}{2}) - (0 - 0)) \\
    &= \frac{1}{16\pi} (0 - (-6) + 6 - 0) = \frac{1}{16\pi} 0 = 0
\end{align*}
\end{proof}

\section*{Exercise IV}

\begin{enumerate}
    \item Show that the Fourier transform of the function
    \[
        f(x) = \frac{1}{x^2 + 1}
    \]
    is
    \[
        \hat{f}(t) = \pi e^{-\abs{t}}
    \]
    
    \begin{proof}
    Let \(g(t) = \pi e^{- \abs{t}}\). Then
    \begin{align*}
        \widehat{g}(x) &= \int_{-\infty}^{+\infty} \pi e^{-\abs{t}} e^{-i t x} \diff t \\
        &= \pi \left(\int_{-\infty}^{0} e^{t} e^{-i t x} \diff t + \int_{0}^{+\infty} e^{-t} e^{-i t x} \diff t\right) \\
        &= \pi \left(\int_{-\infty}^{0} e^{t - i t x} \diff t + \int_{0}^{+\infty} e^{-t - i t x} \diff t\right) \\
        &= \pi \left(\int_{-\infty}^{0} e^{t (1 - i x)} \diff t + \int_{0}^{+\infty} e^{-t (1 + i x)} \diff t\right) \\
        &= \pi \left(\left.\frac{e^{t(1 - i x)}}{1 - i x}\right|_{(-\infty, 0]} - \left.\frac{e^{-t(1 + i x)}}{1 + i x}\right|_{[0, +\infty)}\right) \\
        &= \pi \left(\frac{1}{1 - i x} + \frac{1}{1 + i x} \right) \\
        &= \pi \left(\frac{1 + i x}{1 + x^2} + \frac{1 - i x}{1 + x^2}\right) \\
        &= \pi \frac{1}{1 + x^2}
    \end{align*}
    
    By the inversion formula, we have that
    \[
        g(t) = \pi e^{-\abs{t}} = \frac{1}{2 \pi} \int_{-\infty}^{+\infty} \frac{\pi}{1 + x^2} e^{itx} dx
    \]
    and hence, by making the change \(x = -y\), \(\diff x = - \diff y\), we get that
    \[
        g(t) = \frac{1}{2\pi} \int_{-\infty}^{+\infty} \frac{\pi}{1 + y^2} e^{-ity} \diff y
    \]
    and hence
    \[
        g(t) = \frac{1}{2} \int_{-\infty}^{+\infty} \frac{\pi}{1 + y^2} e^{-ity} \diff y
    \]
    which shows precisely that
    \[
        \widehat{f}(t) = 2 g(t) = 2\pi e^{-\abs{t}}
    \]
    \end{proof}
    
    \item Compute the integral
    \[
        \int_{-\infty}^{+\infty} \frac{\cos x}{x^2 - 4x + 5} \diff x
    \]
    \begin{proof}
    We remark that the integral we're asked to compute is precisely the real part of
    \begin{align*}
        &\int_{-\infty}^{+\infty} \frac{1}{x^2 - 4x + 5} e^{ix} \diff x =
        \int_{-\infty}^{+\infty} \frac{\cos x + i \sin x}{x^2 - 4x + 5} \diff x = \\[1.5ex]
        &= \int_{-\infty}^{+\infty} \frac{\cos x}{x^2 - 4x + 5} \diff x + i \int_{-\infty}^{+\infty} \frac{\sin x}{x^2 - 4x + 5} \diff x
    \end{align*}
    \end{proof}
    
    But note that we also have that, by defining
    \[
        f(x) = \frac{1}{x^2 - 4x + 5}
    \]
    we get that the desired integral is the real part of
    \[
        \widehat{f}(1) = \int_{-\infty}^{+\infty} f(x) e^{i x} \diff x
    \]
    
    Using the change of variable \(x = u + 2\), \(\diff x = \diff u\), we have
    \[
        \widehat{f}(1) = \int_{-\infty}^{+\infty} \frac{1}{u^2 + 1} e^{i (u + 2)} \diff u = \int_{-\infty}^{+\infty} \frac{1}{u^2 + 1} e^{i u} e^{2i} \diff u
    \]
    and with the change of variables \(u = -v\), \(\diff u = - \diff v\), we get
    \[
        \widehat{f}(1) = e^{2i} \int_{-\infty}^{+\infty} \frac{1}{v^2 + 1} e^{-iv} \diff v
    \]
    
    Using the result from the previous exercise, we have that
    \[
        \widehat{f}(1) = e^{2i} \pi e^{-\abs{1}} = e^{2i} \pi \frac{1}{e}
    \]
    of which the real part is
    \[
        \symrm{Re}(\widehat{f}(1)) = \frac{\pi}{e} \cos 2
    \]
\end{enumerate}

\section*{Exercise V}

\begin{enumerate}
    \item Let \(f \in L^1 (\reals)\) be a simple function. Prove that \(\hat{f} \in L^2 (\reals)\).
    \begin{proof}
    Since \(f\) is a simple function, we know that we can write it as
    \[
        f(x) = \sum_{i = 1}^{n} a_i \chi_{A_i} (x)
    \]
    for some collection of disjoint measurable sets \(A_i\) and coefficients \(a_i\).
    
    Now, since \(f\) is in \(L^1 (\reals)\), we know that \(\norm{f}_1 < +\infty\). Hence
    \begin{align*}
        &\int_{-\infty}^{+\infty} \abs{f(x)} \diff x < +\infty \\
        \implies &\int_{-\infty}^{+\infty} \abs*{\sum_{i = 1}^{n} a_i \chi_{A_i} (x)} \diff x < +\infty \\
        \implies &\int_{-\infty}^{+\infty} \left(\sum_{i = 1}^{n} \abs{a_i} \chi_{A_i} (x)\right) \diff x < +\infty
    \end{align*}
    
    Suppose that \(\max_{i} \abs{a_i} = M\). Then, multiplying by \(M\) the inequality above, we get that
    \begin{align*}
        &M \int_{-\infty}^{+\infty} \left(\sum_{i = 1}^{n} \abs{a_i} \chi_{A_i} (x)\right) \diff x < +\infty \\
        \implies &\int_{-\infty}^{+\infty} \left(\sum_{i = 1}^{n} M \abs{a_i}{} \chi_{A_i}\right) \diff x < +\infty \\
        \implies &\int_{-\infty}^{+\infty} \left(\sum_{i = 1}^{n} \abs{a_i}^2 \chi_{A_i}\right) \diff x < +\infty 
    \end{align*}
    where we've used the remark that \(M \abs{a_i} \leq \abs{a_i}^2\).
    
    If we compute the \(L^2\)-norm of \(f\), this is exactly what we need to check:
    \begin{align*}
        \norm{f}_2 &= \sqrt{\int_{-\infty}^{+\infty} \abs{f(x)}^2 \diff x} < +\infty \\
        \iff &\int_{-\infty}^{+\infty} \abs{f(x)}^2 \diff x < +\infty \\
        \impliedby &\int_{-\infty}^{+\infty} \left(\sum_{i = 1}^n \abs{a_i}^2 \chi_{A_i}\right) \diff x < +\infty
    \end{align*}
    where on the last line we rely on the fact that the \(A_i\)s are disjoint, hence \(\chi_{A_i} \cdot \chi_{A_j} = 0\) for \(i \neq j\).
    
    Since \(f \in L^1(\reals) \cap L^2(\reals)\), by Plancherel's theorem we obtain that \(\widehat{f} \in L^2(\reals)\).
    \end{proof}
    
    \item Let \(\varphi_a = \chi_{[-a, a]}\). Find \(f = \varphi_2 \ast \varphi_6\) and \(\hat{f}(2\pi)\).
    \begin{proof}
    We have that
    \begin{align*}
        \varphi_2 \ast \varphi_6 (x) &= \int_{-\infty}^{+\infty} \varphi_2 (t) \varphi_6 (t - x) \diff t \\
        &= \int_{-2}^{2} \varphi_2(t) \varphi_6(t - x) \diff t
    \end{align*}
    since the integrand is \(0\) outside of \([-2, 2] \cap [-6, 6] = [-2, 2]\).
    
    Now, notice that for \(t \in [-2, 2]\),
    \[
        \varphi_6(t - x) = 1 \iff -6 \leq t - x \leq 6 \iff x - 6 \leq t \leq 6 + x
    \]
    
    If \(x \geq 0\), then \(x - 6 \leq - 2 \iff x \leq - 2 + 6 = 4\). Otherwise, if \(x \leq 0\), then \(6 + x \geq 2 \iff x \geq 2 - 6 = -4\).
    
    Hence,
    \[
        f(x) = \varphi_2 * \varphi_6 (x) = \begin{cases}
            4 - x, -2 \leq x \leq 0 \\
            4 + x, 0 \leq x \leq 2 \\
            0, \abs{x} \geq 2
        \end{cases}
    \]
    
    For the second part of the exercise, note that
    \[
        \widehat{f}(t) = \widehat{\varphi_2} (t) \cdot \widehat{\varphi_6} (t)
    \]
    and using the result from the course we get
    \[
        \widehat{f}(t) = \frac{2 \sin (2 t)}{t} \frac{2 \sin (6 t)}{t} = \frac{4 \sin (2t) \sin (6t)}{t^2}
    \]
    
    Hence
    \[
        \widehat{f}(2\pi) = \frac{4 \sin (4 \pi) \sin (12 \pi)}{4\pi^2} = \frac{1}{\pi^2} \cdot 0 \cdot 0 = 0
    \]
    \end{proof}
\end{enumerate}

\end{document}
