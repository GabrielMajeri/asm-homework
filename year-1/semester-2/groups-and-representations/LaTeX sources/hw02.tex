\section*{Homework 2}
\stepcounter{section}

\begin{exercise}
Define the permutations \(a, b, c \in S_6\) by
\begin{align*}
    a = (1 \; 2 \; 3)& &b = (4 \; 5 \; 6)& &c = (2 \; 3) (4 \; 5)
\end{align*}
and let \(G = \generatedby{a, b, c}\).

\begin{enumerate}[(a)]
    \item Check that \(a^3 = b^3 = c^2 = e\), \(ab = ba\), \(c^{-1} a c = a^{-1}\), \(c^{-1} b c = b^{-1}\). Deduce that \(\abs{G} = 18\).
    
    \item Suppose that \(\epsilon, \eta\) are complex cube roots of unity. Prove that there exists a representation \(\rho\) of \(G\) over \(\complex\) such that
    \begin{align*}
        \rho(a) = \begin{pmatrix}
            \epsilon & 0 \\
            0 & \epsilon^{-1}
        \end{pmatrix},
        &
        &\rho(b) = \begin{pmatrix}
            \eta & 0 \\
            0 & \eta^{-1}
        \end{pmatrix},
        &
        &\rho(c) = \begin{pmatrix}
            0 & 1 \\
            1 & 0
        \end{pmatrix}
    \end{align*}
    
    \item For which values of \(\epsilon, \eta\) is \(\rho\) faithful?
    
    \item For which values of \(\epsilon, \eta\) is \(\rho\) irreducible?
\end{enumerate}
\end{exercise}
\begin{solution}
\begin{enumerate}[(a)]
    \item The permutations are written in cycle notation. We have \(\ord(a) = 3\), \(\ord(b) = 3\), \(\ord(c) = \lcm(2, 2) = 2\), hence \(a^3 = b^3 = c^2 = e\).
    
    \(ab\) and \(ba\) commute because they are disjoint cycles.
    
    We also have
    \begin{align*}
        ac &= (1 \; 2 \; 3) ((2 \; 3) (4 \; 5)) = \\
        &= (1 \; 2) (4 \; 5) = \\
        &= (4 \; 5) (1 \; 2) = \\ 
        &= (4 \; 5) (2 \; 3) (3 \; 2 \; 1) \\
        &= ((2 \; 3) (4 \; 5)) (3 \; 2 \; 1) = c a^{-1}
    \end{align*}
    \begin{align*}
        bc &= (4 \; 5 \; 6) ((2 \; 3) (4 \; 5)) = \\
        &= (4 \; 6) (2 \; 3) = \\
        &= (2 \; 3) (4 \; 6) = \\
        &= (2 \; 3) (4 \; 5) (6 \; 5 \; 4) = \\
        &= ((2 \; 3) (4 \; 5)) (6 \; 5 \; 4) = c b^{-1}
    \end{align*}
    where we've used several times the property that disjoint cycles commute.
    
    The relations \(b^{-1} a b = a\), \(c^{-1} a c = a^{-1}\), \(c^{-1} b c = b^{-1}\) show us that \(a\), \(b\) and \(c\) form different conjugacy classes, and hence \(\abs{G} = 3 \cdot 3 \cdot 2 = 18\).
    
    \item To prove that \(\rho\) is a representation, we'll check that the relations defining the group \(G\) hold:
    \[
        \rho(a)^3 = \begin{pmatrix}
            \epsilon & 0 \\
            0 & \epsilon^{-1}
        \end{pmatrix}^3 = \begin{pmatrix}
            \epsilon^3 & 0 \\
            0 & \epsilon^{-3}
        \end{pmatrix} = \begin{pmatrix}
            1 & 0 \\
            0 & 1
        \end{pmatrix} = I_2
    \]
    \[
        \rho(b)^3 = \begin{pmatrix}
            \eta & 0 \\
            0 & \eta^{-1}
        \end{pmatrix}^3 = \begin{pmatrix}
            \eta^3 & 0 \\
            0 & \eta^{-3}
        \end{pmatrix} = \begin{pmatrix}
            1 & 0 \\
            0 & 1
        \end{pmatrix} = I_2
    \]
    \[
        \rho(c)^2 = \begin{pmatrix}
            0 & 1 \\
            1 & 0
        \end{pmatrix}^2 = \begin{pmatrix}
            1 & 0 \\
            0 & 1
        \end{pmatrix} = I_2
    \]
    \[
        \rho(a)\rho(b) = \begin{pmatrix}
            \epsilon & 0 \\
            0 & \epsilon^{-1}
        \end{pmatrix} \begin{pmatrix}
            \eta & 0 \\
            0 & \eta^{-1}
        \end{pmatrix} = \begin{pmatrix}
            \eta & 0 \\
            0 & \eta^{-1}
        \end{pmatrix} \begin{pmatrix}
            \epsilon & 0 \\
            0 & \epsilon^{-1}
        \end{pmatrix} =
        \rho(b) \rho(a)
    \]
    \[
        \rho(c)^{-1}\rho(a)\rho(c) = \begin{pmatrix}
            0 & 1 \\
            1 & 0
        \end{pmatrix} \begin{pmatrix}
            \epsilon & 0 \\
            0 & \epsilon^{-1}
        \end{pmatrix} \begin{pmatrix}
            0 & 1 \\
            1 & 0
        \end{pmatrix} = \begin{pmatrix}
            \epsilon^{-1} & 0 \\
            0 & \epsilon
        \end{pmatrix}
        = \rho(a)^{-1}
    \]
    \[
        \rho(c)^{-1}\rho(b)\rho(c) = \begin{pmatrix}
            0 & 1 \\
            1 & 0
        \end{pmatrix} \begin{pmatrix}
            \eta & 0 \\
            0 & \eta^{-1}
        \end{pmatrix} \begin{pmatrix}
            0 & 1 \\
            1 & 0
        \end{pmatrix} = \begin{pmatrix}
            \eta^{-1} & 0 \\
            0 & \eta
        \end{pmatrix}
        = \rho(b)^{-1}
    \]
    
    \item \(\rho\) is faithful if it is an injective group homomorphism. This happens as long as \(\epsilon\) and \(\eta\) are \emph{distinct} complex cube roots of unity.
    
    \item \(\rho\) is always irreducible since, for example,
    \(\begin{pmatrix}
        \epsilon & 0 \\
        0 & \epsilon^{-1}
    \end{pmatrix}\)
    and
    \(\begin{pmatrix}
        0 & 1 \\
        1 & 0
    \end{pmatrix}\)
    will always have different eigenvectors (for all values of \(\epsilon\)).
\end{enumerate}
\end{solution}

\begin{exercise}
Let \(\symrm{G} = D_3 = \generatedby{r, s \vbar r^3 = s^2 = e, s r = r^{-1} s}\) and \(\omega = e^{\frac{2 \pi i}{3}}\). Prove that the 2-dimensional subspace \(W\) of \(\complex[\symrm{G}]\) defined by
\[
    W = \complex\generatedby{e + \omega^2 r + \omega r^2, s + \omega^2 r s + \omega r^2 s}
\]
is an irreducible \(\complex \symrm{G}\)-submodule of the regular \(\complex \symrm{G}\)-module.
\end{exercise}
\begin{proof}
We will first prove that \(W\) is a \(\complex \symrm{G}\)-submodule of \(\complex[\symrm{G}]\).

Let \(w = \alpha (e + \omega^2 r + \omega r^2) + \beta (s + \omega^2 r s + \omega r^2 s) \in W\). It's enough to check that \(w r \in W\), \(w s \in W\). We have
\begin{align*}
    wr &= (\alpha (e + \omega^2 r + \omega r^2) + \beta (s + \omega^2 r s + \omega r^2 s)) r \\
    &= \alpha (r + \omega^2 r^2 + \omega r^3) + \beta (s r + \omega^2 r s r + \omega r^2 s r) \\
    &= \alpha (\omega e + r + \omega^2 r^2) + \beta (r^{-1} s + \omega^2 r r^{-1} s + \omega r^2 r^{-1} s) \\
    &= \alpha (\omega e + r + \omega^2 r^2) + \beta (\omega^2 s + \omega rs + r^2 s) \\
    &= \alpha \omega (e + \omega^2 r + \omega r^2) + \beta \omega^2 (s + \omega^2 rs + \omega r^2 s) \in W
\end{align*}
\begin{align*}
    ws &= (\alpha (e + \omega^2 r + \omega r^2) + \beta (s + \omega^2 r s + \omega r^2 s))s \\
    &= \alpha (s + \omega^2 rs + \omega r^2 s) + \beta (s^2 + \omega^2 r s^2 + \omega r^2 s^2) \\
    &= \alpha (s + \omega^2 rs + \omega r^2 s) + \beta (e + \omega^2 r + \omega r^2) \in W
\end{align*}

Now we will show that \(W\) is irreducible. Let us denote \(x = e + \omega^2 r + \omega r^2\), \(y = s + \omega^2 r s + \omega r^2 s\) such that \(W = \complex\generatedby{x, y}\).

Suppose that \(W\) were reducible. Since \(\dim_{\complex \symrm{G}} W = 2\), this means we could find a 1-dimensional \(\complex \symrm{G}\)-submodule of \(W\), say \(V = \complex\generatedby{v}\) with \(0 \neq v \in V\) such that \(v r \in V, v s \in V\). This means that \(vr = \lambda v\), \(vs = \lambda' v\), for some \(\lambda, \lambda' \in \complex\).

Since \(v \in W\), we can write it uniquely as \(v = \alpha x + \beta y\) for some \(\alpha, \beta \in \complex\). Then
\[
    v r = (\alpha x + \beta y) r = \alpha \omega x + \beta \omega^2 y
\]
\[
    vs = (\alpha x + \beta y) s = \beta x + \alpha y
\]

Suppose that \(\alpha \omega x + \beta \omega^2 y = \lambda(\alpha x + \beta y)\). Then \(\omega \alpha = \lambda \alpha\), \(\omega^2 \beta = \lambda \beta\). If \(\alpha \neq 0\), then \(\lambda = \omega\), and hence \(\omega^2 \beta = \omega \beta \implies \omega^2 = \omega\), a contradiction. If \(\alpha = 0\), then \(\beta \neq 0\) and \(\lambda = \omega^2\). But now we cannot have \(\beta x + \alpha y = \lambda' (\alpha x + \beta y)\), since this would mean \(\beta x = \lambda' \beta y\). Hence, our assumption that such a submodule \(V\) exists is wrong, thus \(W\) is irreducible.
\end{proof}

\begin{exercise}
Let \(\symrm{G} = C_2 = \generatedby{a \vbar a^2 = e}\).
\begin{enumerate}[1)]
    \item Show that \(f \colon \symrm{F[G]} \to \symrm{F[G]}\) defined by \(f(\alpha e + \beta a) = (\alpha - \beta) (e - a)\) is an \(\symrm{FG}\)-morphism (\(\symrm{F[G]}\) denotes the regular \(\symrm{FG}\)-module).
    
    \item Prove that \(f^2 = 2f\).
    
    \item Find a basis \(B\) of \(\symrm{F[G]}\) such that \([f]_B = \begin{pmatrix}
        2 & 0 \\
        0 & 0
    \end{pmatrix}\).
\end{enumerate}
\end{exercise}
\begin{solution}
\begin{enumerate}[1)]
    \item To prove that \(f\) is a morphism of \(\symrm{FG}\)-modules, we'll show that \(f\) can be written as multiplication by a matrix.
    
    Working in the standard basis of \(\symrm{F[G]}\), we know that
    \begin{align*}
        [f] \cdot \begin{pmatrix} \alpha \\ \beta \end{pmatrix} = \begin{pmatrix}
            \alpha - \beta \\
            \beta - \alpha
        \end{pmatrix}
    \end{align*}
    based on \(f\)'s definition. We therefore deduce that
    \[
        [f] = \begin{pmatrix}
            1 & -1 \\
            -1 & 1
        \end{pmatrix}
    \]
    hence
    \[
        f(\alpha e + \beta a) = \begin{pmatrix}
            1 & -1 \\
            -1 & 1
        \end{pmatrix} \begin{pmatrix}
            \alpha \\
            \beta
        \end{pmatrix}.
    \]
    
    \item Note that \([f^2] = [f \circ f] = [f] \cdot [f]\). The equality becomes
    \[
        [f^2] = \begin{pmatrix}
            1 & -1 \\
            -1 & 1
        \end{pmatrix} \begin{pmatrix}
            1 & -1 \\
            -1 & 1
        \end{pmatrix} =
        \begin{pmatrix}
            2 & -2 \\
            -2 & 2
        \end{pmatrix} =
        2 \begin{pmatrix}
            1 & -1 \\
            -1 & 1
        \end{pmatrix} =
        2 [f]
    \]
    
    \item 
    Notice that \([f]\) is a symmetric matrix, its eigenvalues being \(2\) and \(0\). The corresponding eigenvectors are \(v_1 = \begin{pmatrix}-1 & 1\end{pmatrix}^T\) and \(v_2 = \begin{pmatrix}1 & 1\end{pmatrix}^T\). Hence, if we diagonalize this matrix, we will obtain precisely
    \[
        [f]_B = \begin{pmatrix}
            2 & 0 \\
            0 & 0
        \end{pmatrix}
    \]
    with the basis \(B\) being given by the eigenvectors \(v_1, v_2\).
\end{enumerate}
\end{solution}
