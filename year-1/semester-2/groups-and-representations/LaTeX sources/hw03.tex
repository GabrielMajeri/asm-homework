\section*{Homework 3}
\stepcounter{section}

\begin{exercise}
Let \(G = \generatedby{a \vbar a^3 = e}\) and \(V\) be a \(2\)-dimensional \(\complex\)-vector space with basis \(\Set{ v_1, v_2 }\).
\begin{enumerate}[a)]
    \item Show that \(V\) is a \(\complex \symrm{G}\)-module with \(v_1 a = v_2\), \(v_2 a = - v_1 - v_2\).
    
    \item Express \(V\) as a direct sum of irreducible \(\complex \symrm{G}\)-modules.
\end{enumerate}
\end{exercise}
\begin{proof}
~
\begin{enumerate}[a)]
    \item Writing the relations in the hypothesis in matrix format, we have
    \[
        A \begin{pmatrix}
            1 \\ 0
        \end{pmatrix} =
        \begin{pmatrix}
            0 \\ 1
        \end{pmatrix},
        \quad
        A \begin{pmatrix}
            0 \\ 1
        \end{pmatrix} =
        \begin{pmatrix}
            -1 \\ -1
        \end{pmatrix}.
    \]
    This helps us define the representation
    \[
        A \coloneqq \rho(a) = \begin{pmatrix}
            0 & -1 \\
            1 & -1
        \end{pmatrix}
    \]
    and
    \[
        \rho(a^2) = \rho(a)^2 = A^2 = \begin{pmatrix}
            -1 & 1 \\
            -1 & 0
        \end{pmatrix}
    \]
    Notice that \(\rho(a^3) = A^3 = I_2 = \rho(e)\), which is what we needed.
    
    \item For degree 2 representations, the easiest way to check if they are reducible is to look at their eigenvectors. If the matrices corresponding to the group elements share a common eigenvector, then they are reducible. \\[1ex]
    Both \(A\) and \(A^2\) have the same eigenvalues, \(- \frac{1}{2} + \frac{\sqrt{3}}{2} i\) and \(- \frac{1}{2} - \frac{\sqrt{3}}{2} i\). Hence, this representation of \(G\) is reducible. \\[1ex]
    The two degree 1 subrepresentations of \(\rho\) are
    \begin{align*}
        \rho_1(a^n) &= \left(- \frac{1}{2} + \frac{\sqrt{3}}{2} i\right)^n; \\
        \rho_2(a^n) &= \left(- \frac{1}{2} - \frac{\sqrt{3}}{2} i\right)^n,
    \end{align*}
    corresponding to the \(\complex \symrm{G}\)-submodule structures \(V_1, V_2\) where the right action of \(a\) is multiplication by \(- \frac{1}{2} + \frac{\sqrt{3}}{2} i\), respectively by \(- \frac{1}{2} - \frac{\sqrt{3}}{2} i\).
\end{enumerate}
\end{proof}

\begin{exercise}
Let \(G\) be a finite group and \(\rho \colon G \to GL_2(\complex)\) a representation of \(G\). Suppose that there are \(g, h \in G\) such that \(\rho(g) \rho(h) \neq \rho(h) \rho(g)\). Show that \(\rho\) is irreducible.
\end{exercise}
\begin{proof}
For finite groups, representations of degree two are irreducible iff the matrices \(\Set{ \rho(g) | g \in G }\) don't have any eigenvector in common.

Suppose that \(\rho\) were reducible. This means there exists an eigenvector \(v \neq 0\) and corresponding eigenvalue \(\lambda \in \complex^*\) such that
\[
    \rho(g) v = \lambda v, \forall g \in G.
\]
But then, using the condition from the hypothesis, we obtain
\begin{align*}
    \rho(g) \rho(h) &\neq \rho(h) \rho(g) \\
    \implies
    \rho(g) \rho(h) v &\neq \rho(h) \rho(g) v 
    \tag{\(v\) is non-zero} \\
    \implies
    \rho(g) (\lambda v) &\neq \rho(h) (\lambda v)
    \tag{\(v\) is an eigenvector} \\
    \implies
    \lambda \rho(g) v &\neq \lambda \rho(h) v 
    \tag{multiplication by scalar is commutative} \\
    \implies
    \lambda^2 v &\neq \lambda^2 v
    \tag{\(v\) is an eigenvector} \\
    \implies v &\neq v
    \tag{\(\lambda\) is non-zero}
\end{align*}
We've derived an obvious contradiction, hence our initial assumption must be false; \(\rho\) is irreducible. 
\end{proof}

\begin{exercise}
For \(\symrm{G} = \dihedralgroup{5} = \generatedby{r, s \vbar r^5 = s^2 = e, sr = r^{-1} s}\) and \(\omega = e^{\frac{2 \pi i}{5}} \in \complex\) show that \(\rho \colon \dihedralgroup{5} \to GL_2 (\complex)\) given by
\[
    \rho(r) = \begin{pmatrix}
        \omega & 0 \\
        0 & \omega^{-1}
    \end{pmatrix},
    \quad
    \rho(s) = \begin{pmatrix}
        0 & 1 \\
        1 & 0
    \end{pmatrix}
\]
is a well-defined representation of \(\symrm{G}\).
\end{exercise}
\begin{proof}
It's enough to check that \(\rho(r)^5 = \rho(s)^2 = I_2\), \(\rho(s) \rho(r) = \rho(r)^{-1} \rho(s)\):
\[
    \rho(r)^5 = \begin{pmatrix}
        \omega & 0 \\
        0 & \omega^{-1}
    \end{pmatrix}^5 = \begin{pmatrix}
        1 & 0 \\
        0 & 1
    \end{pmatrix} = I_2
\]
\[
    \rho(s)^2 = \begin{pmatrix}
        0 & 1 \\
        1 & 0
    \end{pmatrix}^2 = \begin{pmatrix}
        1 & 0 \\
        0 & 1
    \end{pmatrix} = I_2
\]
\begin{align*}
    \rho(s) \rho(r) &= \begin{pmatrix}
        0 & 1 \\
        1 & 0
    \end{pmatrix} \begin{pmatrix}
        \omega & 0 \\
        0 & \omega^{-1}
    \end{pmatrix} = \\
    &\qquad = \begin{pmatrix}
        0 & \omega^{-1} \\
        \omega & 0
    \end{pmatrix} = \\
    &= \begin{pmatrix}
        \omega^{-1} & 0 \\
        0 & \omega
    \end{pmatrix} \begin{pmatrix}
        0 & 1 \\
        1 & 0
    \end{pmatrix} = \rho(s) \rho(r)^{-1}
\end{align*}
\end{proof}