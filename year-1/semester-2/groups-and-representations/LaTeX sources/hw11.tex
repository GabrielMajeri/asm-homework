\section*{Homework 11}
\stepcounter{section}

\begin{exercise}
Find the character table of \(\dihedralgroup{4}\).
\end{exercise}
\begin{solution}
A normal subgroup of \(\dihedralgroup{4}\) is its center, which is \(\symrm{H} \coloneqq = \centerofgroup(\dihedralgroup{4}) = \Set{e, r^2}\). Notice that \(\widehat{r} \cdot \widehat{r} = \widehat{e}\), \(\widehat{s} \cdot \widehat{s} = \widehat{e}\), \(\widehat{r} \cdot \widehat{s} = \widehat{rs}\) and \(\widehat{rs} \cdot \widehat{rs} = \widehat{e}\) in the quotient, hence \(\dihedralgroup{4} / \symrm{H}\) is isomorphic to the Klein 4-group.

The character table of the Klein 4-group can be found easily. Being an abelian group, all of its irreducible characters are of degree 1, and the entries must all be roots of unity of degree 2 (i.e.\ \(1\) or \(-1\)):
\begin{center}
    \begin{tabular}{c|c|c|c|c}
        & \(\widehat{e}\) & \(\widehat{r}\) & \(\widehat{s}\) & \(\widehat{rs}\) \\
        \hline
        \(\chi_1\) & 1 & 1 & 1 & 1 \\
        \hline
        \(\chi_2\) & 1 & 1 & -1 & -1 \\
        \hline
        \(\chi_3\) & 1 & -1 & 1 & -1 \\
        \hline
        \(\chi_4\) & 1 & -1 & -1 & 1
    \end{tabular}
\end{center}
This quotient helps us obtain the values of 4 of the 5 irreducible characters of \(\dihedralgroup{4}\).

In order to obtain a degree 2 irreducible representation of \(\dihedralgroup{4}\), we can take \(\rho \colon \symrm{G} \to GL_2 (\complex)\) with
\[
    \rho(r) = \begin{pmatrix}
        i & 0 \\
        0 & -i
    \end{pmatrix},
    \quad
    \rho(s) = \begin{pmatrix}
        0 & 1 \\
        1 & 0
    \end{pmatrix}
\]
whence \(\chi(r) = 0\), \(\chi(s) = 0\).

The character table of \(\dihedralgroup{4}\) is:
\begin{center}
    \begin{tabular}{c|c|c|c|c|c}
        & \(e\) & \(\Set{r^2}\) & \(\Set{r, r^{-1}}\) & \(\Set{s, r^2 s}\) & \(\Set{rs, r^{-1} s}\) \\
        \hline
        \(\chi_1\) & 1 & 1 & 1 & 1 & 1 \\
        \hline
        \(\chi_2\) & 1 & 1 & 1 & -1 & -1 \\
        \hline
        \(\chi_3\) & 1 & 1 & -1 & 1 & -1 \\
        \hline
        \(\chi_4\) & 1 & 1 & -1 & -1 & 1 \\
        \hline
        \(\chi_5\) & 2 & -2 & 0 & 0 & 0
    \end{tabular}
\end{center}
\end{solution}

\begin{exercise}
Let \(\chi_1, \dots, \chi_k\) be the irreducible characters of \(\symrm{G}\). Show that
\[
    \centerofgroup(\symrm{G}) = \Set{ g \in \symrm{G} | \sum_{i = 1}^{k} \chi_i (g) \overline{\chi_i (g)} = \abs{\symrm{G}} }.
\]
\end{exercise}
\begin{proof}
The orthogonality relations for characters tell us that
\[
    \sum_{i = 1}^{k} \chi_i (g_s) \chi_i (g_t) = \delta_{s, t} \abs{\centralizer{\symrm{G}}{g_s}}
\]
or, when \(g_s = g_t\),
\[
    \sum_{i = 1}^{k} (\chi_i (g))^2 = \abs{\centralizer{\symrm{G}}{g}}
\]
\vspace{4pt}

We remark that \(g \in \centerofgroup(\symrm{G})\) iff \(\abs{\centralizer{\symrm{G}}{g}} = \abs{G}\). Thus, the center of \(\symrm{G}\) can be defined as
\[
    \centerofgroup(\symrm{G}) = \Set{ g \in \symrm{G} | \centralizer{\symrm{G}}{g} = \sum_{i = 1}^{k} (\chi_i (g))^2 = \abs{\symrm{G}} }
\]

Because \(z \cdot \overline{z} = \abs{z}^2\) for \(z \in \complex\), the equality from the hypothesis can be restated as
\[
    \sum_{i = 1}^{k} \abs{\chi_i(g)}^2 = \abs{\symrm{G}}
\]
Since \(\abs{\symrm{G}} \in \naturals^*\), this condition is equivalent to \(\sum_{i = 1}^{k} (\chi_i (g))^2 = \abs{\symrm{G}}\).
\end{proof}

\begin{exercise}
Let \(\symrm{G}\) be a finite group and \(\Set{ g_1, \dots, g_k }\) a set of representative elements from the conjugacy classes of \(\symrm{G}\). If \(\chi_1, \dots, \chi_k\) are the irreducible characters of \(\symrm{G}\) and \(A = (\chi_i(g_j))_{1 \leq i, j \leq k}\) the character table of \(\symrm{G}\), show that:
\begin{enumerate}[a)]
    \item \(\det(A)\) is either real or purely imaginary;
    \item \(\abs{\det(A)}^2 = \prod_{i=1}^k \abs{\centralizer{G}{g_i}}\)
\end{enumerate}
\end{exercise}
\begin{proof}
~
\begin{enumerate}[a)]
    \item If we permute the rows or columns of the matrix \(A\), the absolute value of the determinant stays the same but we possibly change the sign.
    
    Let us now replace \(\Set{ g_1, \dots, g_k }\) with \(\Set{ g_1^{-1}, \dots, g_k^{-1} }\). Since \(\chi_i (g_j^{-1}) = \overline{\chi_i (g_j)}\), we obtain the conjugate of the matrix \(A\), whose determinant is \(\overline{\det(A)}\). The resulting matrix is still a permutation of the original character table. Thus
    \[
        \det\left(\overline{A}\right) = \overline{\det(A)} = (-1)^l \det(A)
    \]
    for some \(l \in \naturals\), and this is only possible if \(\det(A)\) is real or purely imaginary.
    
    \item Let \(A^*\) denote the conjugate transpose of the character table, that is \(A^* = \overline{A^T} = (\overline{\chi_j (g_i)})_{1 \leq i, j \leq k}\). Then
    \[
        \det\left(A^*\right) = \det\left(\overline{A^T}\right) = \overline{\det\left(A^T\right)} = \overline{\det(A)}
    \]
    and
    \[
        (A^* A)_{i j} = \sum_{l = 1}^{k} A^*_{i l} A_{l j} = \sum_{l = 1}^{k} \overline{A_{l i}} A_{l j} = \sum_{l = 1}^{k} \chi_l (g_i) \overline{\chi_l (g_j)} = \delta_{i, j} \abs{\centralizer{\symrm{G}}{g_j}}.
    \]
    Thus, \(A^* A\) is a diagonal matrix with entries the orders of the centralizers of the elements of \(\symrm{G}\). 
    
    Since we're working over the complex numbers, we can derive the identity
    \[
        \abs{\det(A)}^2 = \det(A) \cdot \overline{\det(A)} =  \det(A) \cdot \det(A^*) = \det(A^* A)
    \]
    whence
    \[
        \abs{\det(A)}^2 = \prod_{i = 1}^{k} \abs{\centralizer{\symrm{G}}{g_i}}.
    \]
\end{enumerate}
\end{proof}