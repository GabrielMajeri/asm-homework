\section*{Homework 12}
\stepcounter{section}

\begin{exercise}
Use the character table of \(\dihedralgroup{6}\) to find seven distinct normal subgroups of \(\dihedralgroup{6}\).
\end{exercise}
\begin{solution}
We will use the result from the course which describes the character table of \(\dihedralgroup{n}\). Letting \(\omega\) be a root of unity of degree 6, the character table of \(\dihedralgroup{6}\) is:
\begin{center}
    \begin{tabular}{c|c|c|c|c|c|c}
        & \(\Set{e}\) & \(\Set{r^3}\) & \(\Set{r, r^5}\) & \(\Set{r^2, r^4}\) & \(\Set{s, r^2 s, r^4 s}\) & \(\Set{rs, r^3 s, r^5 s}\) \\
        \hline
        \(\chi_1\) & \(1\) & \(1\) & \(1\) & \(1\) & \(1\) & \(1\) \\
        \hline
        \(\chi_2\) & \(1\) & \(1\) & \(1\) & \(1\) & \(-1\) & \(-1\) \\
        \hline
        \(\chi_3\) & \(1\) & \(-1\) & \(-1\) & \(1\) & \(1\) & \(-1\) \\
        \hline
        \(\chi_4\) & \(1\) & \(-1\) & \(-1\) & \(1\) & \(-1\) & \(1\) \\
        \hline
        \(\chi_5\) & \(2\) & \(-2\) & \(\omega + \omega^{-1}\) & \(\omega^2 + \omega^{-2}\) & \(0\) & \(0\) \\
        \hline
        \(\chi_6\) & \(2\) & \(2\) & \(\omega^2 + \omega^{-2}\) & \(\omega + \omega^{-1}\) & \(0\) & \(0\)
    \end{tabular}
\end{center}

Every normal subgroup of \(\dihedralgroup{6}\) is the intersection of the kernels of some characters. Thus, 7 normal subgroups of \(\dihedralgroup{6}\) are:
\begin{itemize}
    \item The whole group \(\dihedralgroup{6}\), the kernel of \(\chi_1\).
    \item The cyclic subgroup \(\Set{e, r, r^2, r^3, r^4, r^5}\), the kernel of \(\chi_2\).
    \item The subgroup \(\Set{e, r^2, r^4, s, r^2 s, r^4 s}\), the kernel of \(\chi_3\).
    \item The subgroup \(\Set{e, r^2, r^4, rs, r^3 s, r^5 s}\), the kernel of \(\chi_4\).
    \item The subgroup \(\Set{e, r^2, r^4}\), the intersection of the kernels of \(\chi_3\) and \(\chi_4\).
    \item The trivial subgroup \(\Set{e}\), the kernel of \(\chi_5\).
    \item The center \(\Set{e, r^3}\), the kernel of \(\chi_6\).
\end{itemize}
\end{solution}

\begin{exercise}
Find the character table of the quaternions,
\[
    \symrm{Q} = \generatedby{ a, b \vbar a^4 = e, a^2 = b^2, b^{-1} a b = a^{-1} }.
\]
\end{exercise}
\begin{solution}
In a previous homework, we've computed the conjugacy classes of \(\symrm{Q}\). They are \(\Set{e}\), \(\Set{a^2}\), \(\Set{a, a^{-1}}\), \(\Set{b, b^{-1}}\) and \(\Set{ab, ba}\).

We remark that \(\symrm{H} = \centerofgroup(\symrm{Q}) = \Set{e, a^2}\) is a normal subgroup of \(\symrm{Q}\), hence we can use the quotient group to obtain a part of the character table for \(\symrm{Q}\). \(\symrm{Q} / \symrm{H}\) is a group of order 4 and all elements have order 2, hence it's isomorphic to the Klein 4-group, for which we computed the character table in a previous homework. This will provide us with four irreducible characters of degree 1.

Since \(\symrm{Q}\) is not commutative, the character table will also contain a character of degree 2. An irreducible representation of \(\symrm{Q}\) of degree 2 is \(\rho \colon \symrm{Q} \to GL_2(\complex)\), defined by
\[
    \rho(a) = \begin{pmatrix}
        i & 0 \\
        0 & -i
    \end{pmatrix},
    \quad
    \rho(b) = \begin{pmatrix}
        0 & 1 \\
        -1 & 0
    \end{pmatrix}
\]

Thus, the character table of the quaternion group is
\begin{center}
    \begin{tabular}{c|c|c|c|c|c}
        & \(\Set{e}\) & \(\Set{a^2}\) & \(\Set{a, a^{-1}}\) & \(\Set{b, b^{-1}}\) & \(\Set{ab, ba}\) \\
        \hline
        \(\chi_1\) & 1 & 1 & 1 & 1 & 1 \\
        \hline
        \(\chi_2\) & 1 & 1 & 1 & -1 & -1 \\
        \hline
        \(\chi_3\) & 1 & 1 & -1 & 1 & -1 \\
        \hline
        \(\chi_4\) & 1 & 1 & -1 & -1 & 1\\
        \hline
        \(\chi_5\) & 2 & -2 & 0 & 0 & 0
    \end{tabular}
\end{center}
\end{solution}

\begin{exercise}
Let \(p\) be a prime number and \(\symrm{G}\) a non-abelian group of order \(p^3\). Show that \(\symrm{G}' = \centerofgroup(\symrm{G})\) (the center of \(\symrm{G}\)).
\end{exercise}
\begin{proof}
Since \(\symrm{G}'\) is a subgroup of \(\symrm{G}\), \(\abs{\symrm{G}'} \in \Set{1, p, p^2, p^3}\). Furthermore, \(\symrm{G}'\) is trivial only when \(\symrm{G}\) is abelian. We know this isn't the case from the hypothesis, hence \(\abs{\symrm{G}'} \in \Set{p, p^2, p^3}\). From the Sylow theorems, there exists a normal subgroup \(\symrm{H}\) of order \(p^2\). Then \(\symrm{G} / \symrm{H}\) is cyclic (hence abelian). This forces \(\abs{\symrm{G}'} \leq p^2\).

The center of \(\symrm{G}\) is also a subgroup of \(\symrm{G}\), hence \(\abs{\centerofgroup(\symrm{G})} \in \Set{1, p, p^2, p^3}\). The center of a \(p\)-group is non-trivial, i.e.\ \(\abs{\centerofgroup(\symrm{G})} > 1\). If \(\abs{\symrm{G} / \centerofgroup(\symrm{G})} = 1\) or \(\abs{\symrm{G} / \centerofgroup(\symrm{G})} = p\), then \(\symrm{G} / \centerofgroup(\symrm{G})\) is cyclic, and a result from the course would imply that \(\symrm{G}\) is abelian, a contradiction with the hypothesis. Hence \(\abs{\centerofgroup(\symrm{G})} = p\).

Now, \(\abs{\symrm{G} / \centerofgroup(\symrm{G})} = p^2\), and this must be an abelian group (reasoning like above, we must have \((\symrm{G} / \centerofgroup(\symrm{G}))' = \Set{e}\), otherwise \(\centerofgroup(\symrm{G} / \centerofgroup(\symrm{G}))\) is nontrivial and \((\symrm{G} / \centerofgroup(\symrm{G})) / \centerofgroup(\symrm{G} / \centerofgroup(\symrm{G}))\) is cyclic). Using a result from the course, we deduce that \(\symrm{G}' \leq \centerofgroup(\symrm{G})\). This forces \(\abs{\symrm{G}'} = p\), and therefore \(\symrm{G}'\) and \(\centerofgroup(\symrm{G})\) are both isomorphic to the cyclic group of order \(p\).
\end{proof}