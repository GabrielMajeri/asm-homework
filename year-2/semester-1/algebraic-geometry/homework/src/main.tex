\documentclass[a4paper, 12pt]{article}

% Better font specification support
\usepackage{fontspec}

% More extensible enumerations
\usepackage{enumerate}

% Commands useful for typesetting mathematics
\usepackage{amsmath}
\usepackage{amsthm}
\usepackage{amssymb}

% Define custom theorem environments for exercises
\theoremstyle{definition}
\newtheorem{exercise}{Exercise}

% Custom solution environment
\newenvironment{solution}
    {\addvspace{8pt}\par\noindent\textit{Solution.}}
    {\hfill\(\square\)}


% Encode mathematical symbols using Unicode characters
\usepackage{unicode-math}

% Support for set-builder notation
\usepackage{braket}

% Many commands for drawing graphics
\usepackage{tikz}
% Support for creating commutative diagrams
\usetikzlibrary{cd}

% Define some useful math commands
\newcommand*{\Spec}[1]{\operatorname{Spec}\left(#1\right)}

\newcommand*{\distinguished}[1]{\symrm{D}\!\left(#1\right)}
\newcommand*{\vanishingset}[1]{\symcal{V}\!\left(#1\right)}

\newcommand*{\reals}{\symbb{R}}
\newcommand*{\complex}{\symbb{C}}

\newcommand*{\finitefield}[1]{\symbb{F}_{#1}}

\newcommand*{\primeideal}[1]{\symfrak{#1}}

\newcommand*{\sheaf}{\symcal{O}}
\newcommand*{\ring}{\symcal{O}}
\newcommand*{\localring}[2]{\ring_{#1 \! , \, #2}}

% Use a nicer default font
\setmainfont{TeX Gyre Schola}
\setmathfont{TeX Gyre Schola Math}

% Conditionally render the solutions
\newif\ifsolutions
\solutionstrue

\begin{document}

\title{Algebraic Geometry II - Homework}
\author{Gabriel Majeri}
% \date{2022--2023}
\date{}

\maketitle

\section*{Homework 1}

The following exercises are from Hartshorne, chapter II, section 2.

\begin{exercise}
Let \(A\) be a ring, \(X = \Spec{A}\), \(f \in A\) and \(\distinguished{f} \subseteq X\) be the open complement of \(\vanishingset{(f)}\). Show that the locally ringed space \(\left(\distinguished{f}), \left.\sheaf_{X}\right|_{\distinguished{f}}\right)\) is isomorphic to \(\Spec{A_{f}}\).
\end{exercise}
\ifsolutions
\begin{solution}
Consider the ring homomorphism
\begin{align*}
    \varphi \colon A &\to A_f \\
    \varphi(a) &= \frac{a}{1}
\end{align*}
This induces a morphism of affine schemes
\begin{align*}
    \overline{\varphi} \colon \Spec{A_f} &\to \Spec{A} \\
    \overline{\varphi} \left(\primeideal{p}\right) &= \varphi^{-1} \left(\primeideal{p}\right)
\end{align*}

We will first prove that the corestriction of this morphism to its image is bijective.

A result from commutative algebra tells us that the prime ideals of the localization \(S^{-1} A\) are precisely the prime ideals of \(A\) which do \textbf{not} intersect the multiplicative system \(S = \Set{ 1, f, f^2, \dots }\). In other words, \(\primeideal{p} \in \Spec{S^{-1} A} = \Spec{A_f} \iff \primeideal{p} \cap S = \emptyset\).

Let \(\primeideal{p} \in \Spec{A}\). The condition \(\primeideal{p} \cap S = \emptyset\) means that \(f^n \not\in \primeideal{p}\), for any \(n\). Since \(\primeideal{p}\) is prime, this is equivalent to \(f \not\in \primeideal{p}\). Passing to ideals, we have that \(\left(f\right) \not\subseteq \primeideal{p}\). This is precisely how we defined the set \(\distinguished{f}\). Thus, \(\overline{\varphi} \colon \Spec{A_f} \to \distinguished{f}\) is bijective.

Now, let us check that \(\overline{\varphi}\) is continuous with respect to the Zariski topology. Since \(\distinguished{f}\) is an open set in \(X\), its subspace topology is generated by elements of the form \(\distinguished{f} \cap \distinguished{g}\), with \(g \in A\). Using the definition, these elements are the same as \(\distinguished{f \cdot g}\). The preimage of such an open set is
\[
\overline{\varphi}^{-1} \left(\distinguished{f \cdot g}\right) = \distinguished{\frac{g}{1}} \subseteq \distinguished{f}
\]
Because \(\overline{\varphi}^{-1}\) returns the generators of the topology on \(\distinguished{f}\) to the generators of the topology on \(\Spec{A_f}\), it is continuous.

We must also check that \(\overline{\varphi}\) induces an isomorphism of sheaves. For any \(\distinguished{f g} \subseteq \distinguished{f}\), define
\begin{align*}
    \overline{\varphi}^{\sharp}_{\distinguished{f g}} \colon \sheaf_{\Spec{A}} \left(\distinguished{f g}\right) &\to \sheaf_{\Spec{A_f}} \left(\distinguished{\frac{g}{1}}\right) \\
    \overline{\varphi}^{\sharp}_{\distinguished{f g}} \left(\frac{a}{f^{n} g^{m}}\right) &= \frac{a/f^n}{g^m}
\end{align*}
This is a ring isomorphism, due to a result from commutative algebra which tells us that successive localizations commute. In particular, looking at the corestriction we have
\[
    \sheaf_{\Spec{\distinguished{f}}} \left(\distinguished{g}\right) \cong \sheaf_{\Spec{A_f}} \left(\distinguished{\frac{g}{1}}\right)
\]
for any \(g \in A\).

These isomorphisms must be compatible with the restriction of regular functions. In other words, the following diagram must commute:
\[
\begin{tikzcd}[row sep=large, column sep=huge]
    \sheaf_{\Spec{A}} \left(\distinguished{f g}\right) \arrow[r, "\overline{\varphi}^{\sharp}_{\distinguished{f g}}"] \arrow[d, "u"] & \sheaf_{\Spec{A_f}} \left(\distinguished{g}\right) \arrow[d, "v"] \\
    \sheaf_{\Spec{A}} \left(\distinguished{f g h}\right) \arrow[r, "\overline{\varphi}^{\sharp}_{\distinguished{f g h}}"] & \sheaf_{\Spec{A_f}} \left(\distinguished{g h}\right)
\end{tikzcd}
\]
where
\begin{align*}
    u \left(\frac{a}{f^n g^m}\right) &= \frac{a h^p}{f^n g^m h^p} \\[0.25em]
    v \left(\frac{a/f^n}{(g/1)^m}\right) &= \frac{(a/f^n) (h/1)^p}{(g/1)^m (h/1)^p}
\end{align*}
The diagram clearly commutes due to the computation rules for fractions.

Finally, we will show that \(\overline{\varphi}\) induces an isomorphism at the level of stalks.

Let \(P \in \distinguished{f}\). Then
\[
    \overline{\varphi}^{\sharp}_{P} \colon \sheaf_{\Spec{A} \! , \, P} \to \sheaf_{\Spec{A_f} \! , \, P}
\]
is an isomorphism of local rings, because we can view the localization at \(P\) as the direct limit of the inductive system of open sets containing \(P\), partially ordered by inclusion and with the restriction maps going between them. The fact that the previous diagram commutes shows that the isomorphism is preserved when taking the limit.
\end{solution}
\fi

\begin{exercise}
Let \(\left(X, \sheaf_{X}\right)\) be a scheme, and let \(U \subseteq X\) be any open subset. Show that \(\left(U, \left.\sheaf_{X}\right|_U\right)\) is a scheme. We call this the \emph{induced scheme structure} on the open set \(U\), and we refer to \(\left(U, \left.\sheaf_{X}\right|_U\right)\) as an \emph{open subscheme} of \(X\).
\end{exercise}
\begin{proof}
Since \(X\) is a scheme, we can cover it with affine open subsets:
\[
    X = \bigcup_{i \in \symcal{I}} X_i
\]
Then
\[
    U = \bigcup_{i \in \symcal{I}} \left(U \cap X_i\right)
\]
where each term \(U \cap X_i\) is itself a scheme. Hence we can write
\[
    U = \bigcup_{i \in \symcal{I}} \left(\, \bigcup_{j \in \symcal{J}\left(\symcal{I}\right)} X_{i j}\right)
\]
where each \(X_{i j}\) is an open affine scheme. This reasoning shows that it's enough to prove the original statement for affine schemes.

Let \(X = \Spec{A}\) be an affine scheme and \(U \subseteq X\) an open subset. Then
\[
    U = \bigcup_{i \in \symcal{I}} \distinguished{f_i}
\]
for some \(f_i \in A\). By the previous exercise, each \(\left(\distinguished{f_i}, \left.\sheaf(X)\right|_{\distinguished{f_i}}\right)\) is an affine scheme, hence their union is a scheme, as claimed.
\end{proof}

\begin{exercise}[Reduced schemes]
A scheme \(\left(X, \sheaf_{X}\right)\) is \emph{reduced} if, for every open set \(U \subset X\), the ring \(\sheaf_X(U)\) has no nilpotent elements.
\begin{enumerate}[(a)]
    \item Show that \(\left(X, \sheaf_{X}\right)\) is reduced if and only if for every \(p \in X\), the local ring \(\localring{X}{p}\) has no nilpotent elements.
    
    \item Let \(\left(X, \sheaf_{X}\right)\) be a scheme. Let \(\left(\sheaf_{X}\right)_{\text{red}}\) be the sheaf associated to the presheaf \(U \mapsto \sheaf_{X}(U)_{\text{red}}\), where for any ring \(A\), we denote by \(A_\text{red}\) the quotient of \(A\) by its ideal of nilpotent elements. Show that \(\left(X, \left(\sheaf_X\right)_{\text{red}}\right)\) is a scheme. We call it the \emph{reduced scheme} associated to \(X\), and denote it by \(X_{\text{red}}\). Show that there is a morphism of schemes \(X_{\text{red}} \to X\), which is a homeomorphism of the underlying topological spaces.
    
    \item Let \(f \colon X \to Y\) be a morphism of schemes, and assume that \(X\) is reduced. Show that there is a unique morphism \(g \colon X \to Y_{\text{red}}\) such that \(f\) is obtained by composing \(g\) with the natural map \(Y_{\text{red}} \to Y\).
\end{enumerate}
\end{exercise}

\setcounter{exercise}{6}

\begin{exercise}
Let \(X\) be a scheme. For any \(x \in X\), let \(\ring_{x}\) be the local ring at \(x\) and \(m_x\) its maximal ideal. We define the \emph{residue field} of \(x\) on \(X\) as the field \(k_x = \ring_x / m_x\). Now let \(K\) be any field. Show that to give a morphism from \(\Spec{K}\) to \(X\) is equivalent to giving a point \(x \in X\) and an inclusion map \(k_x \hookrightarrow K\).
\end{exercise}

\begin{exercise}
Let \(X\) be a scheme. For any point \(x \in X\), we define the \emph{Zariski tangent space} \(T_x\) to \(X\) at \(x\) to be the dual of the \(k_x\)-vector space \(m_x / m_x^2\). Now assume that \(X\) is a scheme over a field \(k\), and let \(k[\epsilon]/(\epsilon^2)\) be the \emph{ring of dual numbers} over \(k\). Show that to give a \(k\)-morphism of \(\Spec{k[\epsilon]/(\epsilon^2)}\) to \(X\) is equivalent to giving a point \(x \in X\), \emph{rational over \(k\)} (i.e. such that \(k_x = k\)), and an element of \(T_x\).
\end{exercise}

\setcounter{exercise}{9}

\begin{exercise}
Describe \(\Spec{\reals[x]}\). How does its topological space compare to the set \(\reals\)? To \(\complex\)?
\end{exercise}

\begin{exercise}
Let \(k = \finitefield{p}\) be the finite field with \(p\) elements. Describe \(\Spec{k[x]}\). What are the residue fields of its points? How many points are there with a given residue field?
\end{exercise}

\end{document}
