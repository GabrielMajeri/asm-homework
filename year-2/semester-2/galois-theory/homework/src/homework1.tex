\section*{Homework 1}

\setcounter{exercise}{0}

\begin{exercise}
Let \(f = X^3 - 2 \in \rationals[X]\). Show that \(\Gal{\splittingfield[\rationals]{f}}{\rationals} \cong S_3\)
\end{exercise}
\begin{solution}
The roots of the polynomial \(f\) are the (complex) cube roots of \(2\), which are \(\sqrt[3]{2}\), \(\omega \sqrt[3]{2}\) and \(\omega^2 \sqrt[3]{2}\), where \(\omega\) is a primitive cube root of unity.

I claim that the splitting field of this polynomial is \(\adjoin{\sqrt[3]{2}, \omega}\). Clearly,
\[
    \adjoin{\sqrt[3]{2}, \omega \sqrt[3]{2}, \omega^2 \sqrt[3]{2}} \subseteq \adjoin{\sqrt[3]{2}, \omega}
\]
For the other inclusion, note that if we take any two distinct roots of \(f\), their quotient is always \(\omega = \omega^{-2}\) or \(\omega^2 = \omega^{-1}\). Dividing by \(\omega\) or one of its powers, we can then recover \(\sqrt[3]{2}\). Thus
\[
    \adjoin{\sqrt[3]{2}, \omega} \subseteq \adjoin{\sqrt[3]{2}, \omega \sqrt[3]{2}, \omega^2 \sqrt[3]{2}}
\]
whence
\[
    \splittingfield{f} = \adjoin{\sqrt[3]{2}, \omega}
\]

An automorphism of \(\splittingfield{f}\) which keeps \(\rationals\) fixed must map the set of algebraic elements \(\Set{ \sqrt[3]{2}, \omega }\) to itself (or to combinations of them). Since \(\left(\sqrt[3]{2}\right)^3 = 2\) and \(\omega^3 = 1\), an automorphism cannot swap these elements. The only possibility is to send \(\sqrt[3]{2}\) to \(\sqrt[3]{2}\), \(\omega \sqrt[3]{2}\) or \(\omega^2 \sqrt[3]{2}\); and to send \(\omega\) to \(\omega\) or \(\omega^2\).

This gives us a total of \(6\) automorphisms, hence \(\abs{\Gal{\splittingfield[\rationals]{f}}{\rationals}} = 6\). Furthermore, looking at the composition table of this group, we observe that it cannot be generated by a single element, hence it must be isomorphic to \(S_3\).
\end{solution}

\begin{exercise}
Let \(g = X^4 - 10 X^2 + 1 \in \rationals[X]\). Show that \(\Gal{\splittingfield{g}}{\rationals} \cong \integersmod{2} \times \integersmod{2}\).
\end{exercise}
\begin{solution}
We must first determine what the splitting field of \(g\) is.

We can explicitly compute the roots of this polynomial. Let \(Y \coloneq X^2\). Then
\begin{gather*}
    Y^2 - 10 Y + 1 = 0 \\
    \iff
    (Y - 5)^2 - 24 = 0 \\
    \iff
    (Y - 5)^2 = 24 \\
    \iff
    Y - 5 = \pm 2 \sqrt{6} \\
    \iff
    Y = 5 \pm 2 \sqrt{6}
\end{gather*}
hence \(X = \pm \sqrt{5 \pm 2 \sqrt{6}}\).

We get that
\[
    \splittingfield{g} = \adjoin{\sqrt{5 + 2 \sqrt{6}}, \sqrt{5 - 2 \sqrt{6}}, -\sqrt{5 + 2 \sqrt{6}}, -\sqrt{5 - 2 \sqrt{6}}}
\]
Since \((-1) \cdot \sqrt{5 \pm 2\sqrt{6}} = -\sqrt{5 \pm 2 \sqrt{6}}\), it's enough to adjoin only the positive roots of each \(Y\):
\[
    \splittingfield{g} = \adjoin{\sqrt{5 + 2 \sqrt{6}}, \sqrt{5 - 2 \sqrt{6}}}
\]
This description of the splitting field is minimal. Indeed, suppose that \(\sqrt{5 + 2 \sqrt{6}}\) were commensurable with \(\sqrt{5 - 2\sqrt{6}}\). We'd get that
\[
    \sqrt{\frac{5 + 2 \sqrt{6}}{5 - 2 \sqrt{6}}} = \frac{p}{q}
\]
for some \(p \in \integers\), \(q \in \naturals^*\), with \((p, q) = 1\). Then
\[
    \frac{5 + 2\sqrt{6}}{5 - 2\sqrt{6}} = \frac{p^2}{q^2}
\]
whence
\begin{gather*}
    q^2 \cdot \left(5 + 2 \sqrt{6}\right) = p^2 \cdot \left(5 - 2 \sqrt{6}\right) \\
    \implies 5 q^2 + 2 q^2 \sqrt{6} = 5 p^2 - 2 p^2 \sqrt{6} \\
    \implies 2 (p^2 + q^2) \sqrt{6} = 5 (p^2 - q^2) \\
    \implies \sqrt{6} = \frac{5 (p^2 - q^2)}{2 (p^2 + q^2)}
\end{gather*}
contradicting the irrationality of \(\sqrt{6}\).

Let \(\alpha \coloneq \sqrt{5 + 2 \sqrt{6}}\), \(\beta \coloneq \sqrt{5 - 2 \sqrt{6}}\). A field automorphism which keeps \(\rationals\) fixed must map each root to itself (or to its negative) or to the other root (or its negative). In this field extension we also have the identity \(\alpha \cdot \beta = 1\). Hence, we can't (for example) map \(\alpha\) to itself and \(\beta\) to \(-\beta\), since that would change this equality to \(\alpha \cdot \beta = -1\); once we choose a sign for one root, we must use the same sign for the other root as well. This leads us to conclude that there are four possible morphisms:
\begin{align*}
    \alpha \mapsto \alpha, \beta \mapsto \beta & & \alpha \mapsto -\alpha, \beta \mapsto -\beta \\[1em]
    \alpha \mapsto \beta, \beta \mapsto \alpha & & \alpha \mapsto -\beta, \beta \mapsto -\alpha
\end{align*}
We can see that the composition rules for these morphisms match up to the structure of the group \(\integersmod{2} \times \integersmod{2}\).
\end{solution}

\begin{exercise}
Let \(h = X^4 + 1 \in \rationals[X]\). Show that \(h \in \finitefield{p}[X]\) is reducible for any prime \(p\), although \(h\) is irreducible in \(\rationals[X]\).
\end{exercise}
\begin{solution}
The roots of \(h \in \rationals[X]\) are the primitive fourth roots of unity. These are \(\cos \frac{(2k + 1) \pi}{4} + i \sin \frac{(2k + 1) \pi}{4}\) for \(k \in \Set{0, 1, 2, 3}\), and they are not rational. Hence, \(h\) cannot be factored by a degree one polynomial in \(\rationals[X]\).

Is it possible for \(h\) to be the product of two irreducible polynomials of degree two? We would have
\begin{align*}
    X^4 + 1 &= (X^2 + a X + b) (X^2 + c X + d) \\
    &= X^4 + a X^3 + c X^3 + b X^2 + d X^2 + a c X^2 + \\
    & \hspace{4em} + a d X + b c X + b d \\
    &= X^4 + (a + c) X^3 + (b + d + ac) X^2 \\
    & \hspace{4em} + (ad + bc) X + bd
\end{align*}
Comparing the two sides of the equation term by term, we get
\[
\begin{cases}
    a + c = 0 \\
    b + d + ac = 0 \\
    ad + bc = 0 \\
    bd = 1
\end{cases}
\implies
\begin{cases}
    a = -c \\
    b + d = c^2 \\
    c (b - d) = 0 \\
    bd = 1
\end{cases}
\]
We're left with two possibilities:
\begin{itemize}
    \item \(c = 0\), which implies \(a = 0\) and \(b = -d\). But then we reach \(b^2 = -1\), which cannot be solved over \(\rationals\).

    \item \(b = d\), which implies \(b^2 = 1\), whence \(b \in \Set{ 1, -1 }\). But this contradicts \(2b = c^2\), since \(c^2\) cannot equal \(2\) or \(-2\) over \(\rationals\).
\end{itemize}
This analysis concludes that \(h\) is irreducible over \(\rationals[X]\).

When working over the finite field \(\finitefield{p}\), we can consider the same system of equations as above. There are once again two cases:
\begin{itemize}
    \item If \(-1\) is a quadratic residue modulo \(p\), then there exists some \(n\) such that \(n^2 \equiv -1 \Mod{p}\); in this case, we have \((X^2 + n) (X^2 - n) = X^4 - n^2 = X^4 + 1 = h\).

    \item If \(-1\) is not a quadratic residue modulo \(p\), then we know from number theory that \(p \equiv 3 \Mod{4}\). Furthermore, if \(p \equiv 7 \Mod{8}\), then \(2\) is a quadratic residue, hence we can take \(b = d = 1\) and find \(c\) satisfying \(c^2 \equiv 2 \Mod{p}\). Otherwise, using the properties of the Legendre symbol we deduce that we can take \(b = d = -1\) and find \(c\) satisfying \(c^2 \equiv -2 \Mod{p}\).
\end{itemize}
In both cases, we can factor \(h\) as the product of two irreducible polynomials of degree 2.
% TODO
\end{solution}