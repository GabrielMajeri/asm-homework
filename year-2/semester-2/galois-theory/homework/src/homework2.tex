\section*{Homework 2}

\setcounter{exercise}{0}

\begin{exercise}
Let \(f = X^m - 1 \in k[X]\), \(m \in \naturals^*\). Let \(E = k_f\). Show that \(\Gal{E}{k}\) is isomorphic to a subgroup of \(\units{\integersmod{m}}\).
\end{exercise}
\begin{solution}
The roots of \(f\) are the \(m\)th roots of unity. We will assume known the fact that these form a finite cyclic subgroup of \(K^{\times}\), where \(K\) is the algebraic closure of \(k\). We will denote the group of \(m\)th roots of unity by \(\rootsofunity{m}\). An \(m\)th root of unity is called \emph{primitive} if it is a generator of this group.

Let \(\sigma \in \Gal{E}{k}\) and let \(\zeta\) be a primitive root of unity. We will show that \(\sigma(\zeta)\) is a primitive root of unity as well. Indeed, since \(\sigma\) is multiplicative, we must have \(\sigma(\zeta)^m = \sigma\left(\zeta^m\right) = \sigma(1) = 1\), which shows that \(\sigma(\zeta)\) is an \(m\)th root of unity. Furthermore, we cannot have \(\sigma(\zeta)^j = 1\) for any \(1 \leq j < m\), since then we'd get \(\sigma\left(\zeta^j\right) = 1\), which would imply \(\zeta^j = 1\), contradicting the primitivity of \(\zeta\). Hence \(\sigma(\zeta)\) is also primitive.

For a fixed \(\sigma\), we claim that there exists \(n_{\sigma} \in \naturals^*\) such that \(\sigma(r) = r^{n_{\sigma}}\), for all \(r \in \rootsofunity{m}\). Fix a primitive root of unity \(\zeta\), write every element of \(\rootsofunity{m}\) as \(\zeta^l\) for some \(l \in \naturals^*\), and take \(n\) to be the exponent \(l\) corresponding to the element \(\sigma(\zeta)\). Notice that the choice of these integers is unique modulo \(m\).

We will define a group homomorphism \(\Phi \colon \Gal{E}{k} \to \units{\integersmod{m}}\) by mapping every \(\sigma \in \Gal{E}{k}\) to the \(n \Mod{m}\) we've described above. For \(\sigma, \sigma' \in \Gal{E}{k}\) and a primitive \(m\)th root of unity \(\zeta\), we have that
\[
    (\sigma \circ \sigma') (\zeta) = \sigma (\sigma' (\zeta)) = \sigma \left(\zeta^{n_{\sigma'}}\right) = \left(\zeta^{n_{\sigma'}}\right)^{n_{\sigma}} = \zeta^{n_{\sigma} n_{\sigma'}} = \tau(\sigma)
\]
where \(n_{\tau} \equiv n_{\sigma} n_{\sigma'} \Mod{m}\). Hence
\[
    \Phi(\sigma \circ \sigma') = \Phi(\sigma) \Phi(\sigma')
\]
For injectivitiy, note that \(n_{\sigma} \equiv 1 \Mod{m}\) implies that \(\sigma(\zeta) = \zeta\), and since \(\zeta\) is a generator of \(\rootsofunity{m}\), this shows that \(\sigma = \symrm{Id}_{\Gal{E}{k}}\). Thus, \(\Phi\) embeds \(\Gal{E}{k}\) into \(\units{\integersmod{m}}\).
\end{solution}

\begin{exercise}
For \(p\) prime, prove that \(\Gal{\finitefield{p^n}}{\finitefield{p}} \cong \integersmod{n}\) and that the generator is the Frobenius morphism, \(\Frob \colon \finitefield{p^n} \to \finitefield{p^n}\), \(\Frob(u) = u^p\).
\end{exercise}
\begin{solution}
We remark that \(\degree{\finitefield{p^n}}{\finitefield{p}} = n\) and that this extension is Galois, by the construction of finite fields. If we can prove that the Frobenius map is an automorphism and that it has degree \(n\), then \(\Gal{\finitefield{p^n}}{\finitefield{p}}\) will be generated by it.

% Proof based on https://math.stackexchange.com/q/637918/388180
\begin{itemize}
    \item \(\Frob\) is an \textbf{endomorphism}. Since we're working in characteristic \(p\), we have
    \[
        \Frob(a + b) = (a + b)^p = a^p + b^p = \Frob(a) + \Frob(b)
    \]
    and by the associativity and commutativity of multiplication
    \[
        \Frob(a \cdot b) = (a \cdot b)^p = a^p \cdot b^p = \Frob(a) \cdot \Frob(b)
    \]
    
    \item \(\Frob\) is an \textbf{automorphism}. We remark that
    \[
        \Frob(a) = 0 \implies a^p = 0 \implies a = 0,
    \]
    since fields don't have zero divisors (besides \(0\) itself). This shows that \(\Frob\) is \textbf{injective}. Since the domain and codomain are the same finite set, by the pigeonhole principle we conclude that \(\Frob\) is also \textbf{surjective}.

    \item \(\Frob\) has \textbf{order \(n\)}. Since \(\degree{\finitefield{p^n}}{\finitefield{p}} = n\), \(\Frob\) has degree \emph{at most} equal to \(n\). Suppose that \(\Frob\) has degree \(k < n\), i.e.
    \[
        \Frob^k = \Id \iff a^{p^k} = a, \forall a \in \finitefield{p^n}
    \]
    But then we'd get that the polynomial \(X^{p^k} - X \in \finitefield{p^n}[X]\) has more than \(p^k\) distinct roots, a contradiction. Hence \(\ord \Frob = n\).
\end{itemize}

The previous statements show that
\[
    \Gal{\finitefield{p^n}}{\finitefield{p}} = \generatedby{ \Frob \vbar \Frob^n = \Id } \cong \integersmod{n}
\]
\end{solution}

\begin{exercise}
Let \(f \in k[X]\) be a separable polynomial and let \(E = k_f\). Then \(f\) is irreducible iff \(\forall \alpha, \beta \in E\) roots of \(f\), there exists \(\sigma \in \Gal{E}{k}\) such that \(\sigma(\alpha) = \beta\).
\end{exercise}
\begin{proof}
~
\begin{itemize}
    \item[\(\implies\)] Assume that \(f\) is irreducible. The conclusion is obviously true if \(\deg \, f = 1\) and \(k_f = k\). Suppose that \(n \coloneq \deg \, f > 1\). Then we can obtain \(k_f\) by adjoining any of the roots \(\alpha_1, \dots, \alpha_n\) of \(f\) to \(k\). Hence, we have the chain of isomorphisms
    \[
        k_f = k[X]/(f) \cong \adjoin[k]{\alpha_1} \cong \dots \cong \adjoin[k]{\alpha_n}
    \]
    Let \(\sigma\) be the field isomorphism which takes \(\adjoin[k]{\alpha_i}\) to \(\adjoin[k]{\alpha_j}\). It must be the case that \(\sigma\) fixes \(k\), hence \(\sigma \in \Gal{E}{k}\) and \(\sigma\left(\alpha_i\right) = \alpha_j\).

    \item[\(\impliedby\)] We will prove the reverse implication by contradiction. Suppose that \(f\) were reducible and separable. Then \(f = g \cdot h\) for some polynomials \(g, h \in k[X]\). If \(\deg g = \deg h = 1\), then \(\degree{E}{k} = 1\), in which case there is nothing to prove. Suppose that \(\deg g > 1\). Then \(g\) and \(h\) do not have any roots in common (otherwise, \(f\) wouldn't be separable). In \(k_g\), we have that
    \[
        f = g_1 \cdot \hdots \cdot g_l \cdot h_1 \cdot \hdots \cdot h_m
    \]
    with \(l > 1\) and \(m \geq 1\) (it could be the case that \(h\) doesn't split at all in \(k_g\)). A field isomorphism \(\sigma \in \Gal{k_g}{k}\) will permute the \(g_i\)s, but won't affect the \(h_j\)s. Furthermore, since \(k \subseteq k_g \subseteq k_f\) and analogously \(k \subseteq k_h \subseteq k_f\), any field isomorphism of \(k_f\) must extend one of these isomorphisms. But this shows that there is no \(\sigma \in \Gal{E}{k}\) which would map a root of \(g\) to a root of \(h\), or vice-versa.
\end{itemize}
\end{proof}
