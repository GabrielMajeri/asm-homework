\section*{Homework 5}

\setcounter{exercise}{0}

\begin{exercise}
Let \(S\) be a finite set and \(k[S]\) the \(k\)-vector space having \(S\) as a basis, i.e.
\[
    k[S] = \Set{ \sum_{s \in S} \alpha_s s | \alpha_s \in k, \forall s \in S }
\]
Show that \(k[S]\) is a Hopf algebra with comultiplication \(\Delta\) and counit \(\varepsilon\) given by
\[
\begin{cases}
    \Delta(s) = s \tensor s \\
    \varepsilon(s) = 1
\end{cases}
\]
if and only if \(S\) is a group.
\end{exercise}
\begin{proof}
We will prove the statement by double implications.
\begin{itemize}
    \item[\(\implies\)] Suppose \(k[S]\) is a Hopf algebra. Denote by \(G\) the set of all elements \(g\) of \(H\) for which \(\Delta(g) = g \tensor g\) and \(\varepsilon(g) = 1\).

    Let \(g, h \in G\). Then
    \begin{gather*}
        \Delta(g) \Delta(h) = (g \tensor g) (h \tensor h) = gh \tensor gh = \Delta(gh) \\
        \varepsilon(gh) = \varepsilon(g) \varepsilon(h) = 1 \cdot 1 = 1
    \end{gather*}
    Hence, \(G\) is closed under multiplication.

    Let \(g \in G\). By the properties of the Hopf algebra antipode, we have
    \[
        S(g) \cdot g = \varepsilon(g) \cdot 1 \implies S(g) \cdot g = 1
    \]
    This shows that \(S(g)\) is the multiplicative inverse of \(g\).
    
    Applying \(\Delta\) to the above relation, we obtain
    \[
        \Delta(S(g)) \cdot (g \tensor g) = 1 \tensor 1
    \]
    which shows that \(\Delta(S(g))\) is the multiplicative inverse of \(g \tensor g\) in \(k[S] \tensor k[S]\). By uniqueness of inverses, we get that \(\Delta(S(g)) = S(g) \tensor S(g)\). Furthermore, we've shown in a previous homework that \(\varepsilon(S(g)) = \varepsilon(g) = 1\), which lets us conclude that \(S(g) \in G\) as well.

    We've shown that the set \(G\) is a group with respect to the multiplication induced from the \(k\)-algebra structure on \(k[S]\). Notice that \(S \subseteq G \subseteq k[S]\) and that \(k[S]\) is generated by \(S\) as a \(k\)-vector space, hence \(S = G\).

    \item[\(\impliedby\)] Suppose that \(S\) is a group.

    Let \(s \in S\). We have:

    \begin{itemize}
        \item \(\Delta\) is coassociative:
        \begin{gather*}
            (\Delta \tensor \Id) \Delta (s) = (\Delta \tensor \Id) (s \tensor s) = \\
            = (s \tensor s) \tensor s = s \tensor (s \tensor s) = \\
            = (\Id \tensor \Delta) (s \tensor s) = (\Id \tensor \Delta) \Delta (s)
        \end{gather*}

        \item \(\varepsilon\) is counit for \(\Delta\):
        \begin{gather*}
            (\varepsilon \tensor \Id) \Delta (s) = (\varepsilon \tensor \Id) (s \tensor s) = (\varepsilon(s) \tensor s) = 1 \tensor s \\
            (\Id \tensor \varepsilon) \Delta (s) = (\Id \tensor \varepsilon) (s \tensor s) = (s \tensor \varepsilon(s)) = s \tensor 1
        \end{gather*}

        \item \(S\) is an antipode, where \(S(s) = s^{-1}\): \begin{gather*}
            \eta \varepsilon (s) = \eta (1) = 1 \\
            m_H (S \tensor \Id) \Delta (s) = m_H (S \tensor \Id) (s \tensor s) = m_H (S(s) \tensor s) = S(s) \cdot s = 1 \\
            m_H (\Id \tensor S) \Delta (s) = m_H (\Id \tensor S) (s \tensor s) = m_H (s \tensor S(s)) = s \cdot S(s) = 1
        \end{gather*}
    \end{itemize}
\end{itemize}
\end{proof}

\begin{exercise}
Let \(\symrm{SL}_n (k) = \Set{ A \in M_n(k) | \det A = 1 }\) and regard it as \(\symrm{SL}_n (k) \subset k^{n^2}\). Show that \(\symrm{SL}_n (k)\) is an affine algebraic group and describe the Hopf algebra structure of \(k\left[\symrm{SL}_n (k)\right]\).
\end{exercise}
\begin{proof}
From linear algebra, we know that the determinant of a matrix is a polynomial in the matrix coefficients. For a fixed \(n \in \naturals^*\), let \(D_n \left(a_{1, \, 1}, \dots, a_{n, \, n}\right)\) be the polynomial corresponding to the determinant of the matrix
\[
    A = \begin{pmatrix}
        a_{1, \, 1} & a_{1, \, 2} & \hdots & a_{1, \, n} \\
        a_{2, \, 1} & a_{2, \, 2} & \hdots & a_{2, \, n} \\
        \vdots & \vdots & \ddots & \vdots \\
        a_{n, \, 1} & a_{n, \, 2} & \dots & a_{n, \, n}
    \end{pmatrix}
\]
We can view \(\symrm{SL}_n (k)\) as the set of common zeros of the polynomial \(D_n\). Hence, \(\symrm{SL}_n (k)\) is an affine algebraic set.

Furthermore, this affine variety is also a group. For \(A, B \in \symrm{SL}_n (k)\), let \(A \cdot B\) denote the usual matrix product. Since
\[
    \det(A \cdot B) = \det(A) \cdot \det(B) = 1 \cdot 1 = 1
\]
we have \(A \cdot B \in \symrm{SL}_n (k)\). This is a regular map, since the element at row \(i\) and column \(j\) of the result is given by a homogeneous polynomial of degree two,
\[
    c_{i, \, j} = \sum_{t = 1}^{n} a_{i, \, t} \cdot b_{t, \, j}
\]
The same is true for the unit map \(i \colon \Set{e} \hookrightarrow \symrm{SL}_n (k)\), \(i(e) = \Id_n\) and the inverse map \(A \mapsto A^{-1}\), since we can compute the matrix inverse using the adjugate matrix method. The entries of the adjugate are cofactors, i.e.\ a sign term times minors of the matrix, and the minors are determinants of matrices of size \(n - 1\), hence they are regular maps.

Since \(\symrm{SL}_n (k)\) is a group, the coordinate ring \(k\left[\symrm{SL}_n (k)\right]\) (the ring of regular functions on \(\symrm{SL}_n (k)\)) has the structure of a Hopf algebra.

For any \(f \in k\left[\symrm{SL}_n (k)\right] = \Hom_k \left(\symrm{SL}_n (k), k\right)\), we have:
\begin{gather*}
    \Delta\left(\, f\right) = f \tensor f \\
    \varepsilon\left(\, f\right) = 1 \\
    S\left(\, f(A)\right) = \left(f(A)\right)^{-1}
\end{gather*}
\end{proof}

\begin{exercise}
For \(H\) a finite dimensional Hopf \(k\)-algebra and \(k \subseteq E\) a field extension, show that \(k \subseteq E\) is \(H^*\)-Galois if and only if \(E\) is an \(H\)-module algebra and
\begin{align*}
    \mu \colon E \tensor H &\to \End_k (E) \\
    x \tensor h &\mapsto (y \mapsto x (h \cdot y))
\end{align*}
is bijective (where \(\, \cdot \, \colon H \tensor E \to E\) is the action of \(H\) on \(E\)).
\end{exercise}
\begin{proof}
In the course, we've proved Artin's theorem, which states that \(k \subseteq E\) is \(H^*\)-Galois if and only if the map
\begin{gather*}
    \beta \colon E \tensor_{k} E \to E \tensor_{k} H^* \\[0.5em]
    \beta\left(e \tensor_{k} f\right) = \sum_{i} e \left(h_i \cdot f\right) \tensor h^i
\end{gather*}
is bijective, where \(\Set{ h_i, h^i }\) are dual bases in \(H\) and \(H^*\).

In the previous homework, we've shown that there exists an isomorphism
\[
    \Hom_k \left(E \tensor_k E, E \tensor_k V^*\right) \cong \Hom_k \left(E \tensor_k V, \End_k(E)\right)
\]
which takes bijective maps to bijective maps. In particular, setting \(V = H\), we have
\[
    \Hom_k \left(E \tensor_k E, E \tensor_k H^*\right) \cong \Hom_k \left(E \tensor_k H, \End_k(E)\right)
\]
This isomorphism can be used to convert the map \(\beta \in \Hom_k \left(E \tensor_k E, E \tensor_k H^*\right)\) into the map \(\mu \in \Hom_k \left(E \tensor_k H, \End_k(E)\right)\) and vice-versa. Thus, the two conditions we have for the extension \(k \subseteq E\) to be \(H^*\)-Galois are in fact equivalent.
\end{proof}