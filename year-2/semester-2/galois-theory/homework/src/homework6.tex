\section*{Homework 6}

\setcounter{exercise}{0}

\begin{exercise}
Let \(H_N\) be the ``co''-version of the Hopf algebra \(H_n\) described in exercise 2 from Homework 4. As an algebra, it is generated by \(X\), \(Y\) with the relations \(X^N = 1\), \(Y^N = 0\) and \(Y X = \omega X Y\) (\(\omega\) is an \(N^{\text{th}}\)-primitive root of unity in \(k\)). The coalgebra structure is given by
\[
\begin{cases}
    \Delta(X) = X \tensor X, \varepsilon(X) = 1 \\
    \Delta(Y) = 1 \tensor Y + Y \tensor X, \varepsilon(Y) = 0
\end{cases}
\]
and the antipode \(S\) is determined by \(S(X) = X^{-1}\) and \(S(Y) = - \omega^{-1} X^{-1} Y\), extended as an anti-algebra homomorphism.

For \(\alpha, \beta \in k\) show that:
\begin{enumerate}[(i)]
    \item \(A_{\alpha, \beta} \coloneq \frac{k \generatedby{x, y}}{\generatedby{x^N - \alpha, \, y^N - \beta, \, yx - \omega xy}}\) is an \(H_N\)-comodule algebra via the structure defined by
    \[
    \begin{cases}
        \rho(x) = x \tensor X\\
        \rho(y) = 1 \tensor Y + y \tensor X
    \end{cases}
    \]

    \item \(A_{\alpha, \beta}^{\symrm{co} \left(H_N\right)} \cong k\) and \(k \subseteq A_{\alpha, \beta}\) is \(H_N\)-Galois.
\end{enumerate}
\end{exercise} 
\begin{proof}
~
\begin{enumerate}[(i)]
    \item We remark that \(A_{\alpha, \beta}\) is an associative \(k\)-algebra, since it is a quotient of the free \(k\)-algebra on two generators. In order for it to be an \(H_N\)-comodule, the \(k\)-linear map \(\rho \colon A_{\alpha, \beta} \to A_{\alpha, \beta} \tensor H_N\) given in the hypothesis must verify the following two conditions:
    \begin{itemize}
        \item \((\Id \tensor \Delta) \rho = (\rho \tensor \Id) \rho\). We can check this on the generators of \(A_{\alpha, \beta}\). For \(x\):
        \begin{gather*}
            (\Id \tensor \Delta) \rho (x) = (\Id \tensor \Delta) (x \tensor X) = x \tensor X \tensor X \\[0.5em]
            (\rho \tensor \Id) \rho (x) = (\rho \tensor \Id) (x \tensor X) = \rho(x) \tensor X = x \tensor X \tensor X
        \end{gather*}
        and for \(y\):
        \begin{gather*}
            (\Id \tensor \Delta) \rho (y) = (\Id \tensor \Delta) (1 \tensor Y + y \tensor X) \\
            = (\Id \tensor \Delta) (1 \tensor Y) + (\Id \tensor \Delta) (y \tensor X) \\
            = 1 \tensor \Delta(Y) + y \tensor \Delta(X) \\
            = 1 \tensor (1 \tensor Y + Y \tensor X) + y \tensor (X \tensor X) \\
            = 1 \tensor 1 \tensor Y + 1 \tensor Y \tensor X + y \tensor X \tensor X \\[0.5em]
            (\rho \tensor \Id) \rho (y) = (\rho \tensor \Id) (1 \tensor Y + y \tensor X) \\
            = \rho(1) \tensor Y + \rho(y) \tensor X \\
            = 1 \tensor 1 \tensor Y + (1 \tensor Y + y \tensor X) \tensor X \\
            = 1 \tensor 1 \tensor Y + 1 \tensor Y \tensor X + y \tensor X \tensor X
        \end{gather*}
        
        \item \((\Id \tensor \varepsilon) \rho = \Id\). We can check this on generators as well. For \(x\):
        \[
            (\Id \tensor \varepsilon) \rho(x) = (\Id \tensor \varepsilon) (x \tensor X) = x \tensor \varepsilon(X) = x \tensor 1
        \]
        and for \(y\):
        \begin{gather*}
            (\Id \tensor \varepsilon) \rho(y) = (\Id \tensor \varepsilon) (1 \tensor Y + y \tensor X) \\
            = (\Id \tensor \varepsilon) (1 \tensor Y) + (\Id \tensor \varepsilon) (y \tensor X) = (1 \tensor \varepsilon(Y)) + (y \tensor \varepsilon(X)) \\
            = 1 \tensor 0 + y \tensor 1 = y \tensor 1
        \end{gather*}
    \end{itemize}
    This shows that \(A_{\alpha, \beta}\) is an \(H_N\)-comodule algebra.

    \item The coinvariants of \(A_{\alpha, \beta}\) are those elements \(a \in A_{\alpha, \beta}\) for which
    \[
        \rho(a) = a \tensor 1
    \]

    In the case when \(\alpha = 1\) and \(\beta = 0\), \(A_{\alpha, \beta}\) is isomorphic to \(H_N\) as an algebra. We have
    \[
        \rho\left(x^N\right) = x^N \tensor X^N = 1 \tensor 1 = \rho(1)
    \]
    and
    \[
        \rho\left(y^N\right) = 1 \tensor Y^N + y^N \tensor X^n + \omega^N y^N \tensor XY^{N - 1} = 0 = 0 \tensor 0 = \rho(0)
    \]
    whence
    \[
        A_{\alpha, \beta}^{\symrm{co} \left(H_N\right)} \cong H_N
    \]
    and this (trivial) extension is clearly Hopf Galois.

    Otherwise,
    \[
        \rho\left(x^N\right) = \alpha \tensor 1 \neq \alpha \tensor \alpha = \rho(\alpha)
    \]
    and similarly for \(y^N\). Other powers of \(x\) and \(y\) are not coinvariant either. Hence,
    \[
        A_{\alpha, \beta}^{\symrm{co} \left(H_N\right)} \cong k
    \]

    Using the definition from the course, the extension \(k \subseteq A_{\alpha, \beta}^{\symrm{co} \left(H_N\right)}\) is \(H_N\)-Galois iff the map
    \begin{gather*}
        \beta \colon A_{\alpha, \beta} \tensor_{k} A_{\alpha, \beta} \to A_{\alpha, \beta} \tensor_{k} H_N \\[0.5em]
        \beta\left(a \tensor_{k} b\right) = a b_{(0)} \tensor b_{(1)}
    \end{gather*}
    is bijective, where \(b_{(0)}\) and \(b_{(1)}\) is the Sweedler notation for the components of \(\rho(b)\).

    We remark that, by construction, \(A_{\alpha, \beta}\) and \(H_N\) have the same dimension as \(k\)-vector spaces. Hence, by the rank-nullity theorem, it is enough to show that \(\beta\) is surjective.

    First, we have \(\beta(a \tensor 1) = a \tensor 1\) for every \(a \in A_{\alpha, \beta}\). We also find easily that \(\beta(a \tensor x) = a x \tensor X\). Taking \(a = x^{N - 1}\), we find that \(1 \tensor X\) is in the image (hence also its linear span). For \(a = y x^{N - 1}\) we get that \(y \tensor X\) is in the image.

    Evaluating \(\beta(1 \tensor Y)\) gives us \(1 \tensor Y + y \tensor X\). Subtracting \(y \tensor X\), we get that \(1 \tensor Y\) is in the image by linearity.
    
    Repeating the arguments above for higher powers of \(X\) and \(Y\), we get that the image of \(\beta\) is the whole of \(A_{\alpha, \beta} \tensor H_N\), hence it is bijective.
\end{enumerate}
\end{proof}

\begin{exercise}
Let \(C_n = \generatedby{ g \vbar g^n = e }\) and \(A = \frac{k[X]}{(X^n - a)} = \adjoin[k]{\zeta}\), \(\zeta^n = a \in k\). Show that \(k \subseteq A\) is \(k[C_n]\)-Galois (in the Hopf sense).
\end{exercise}
\begin{proof}
Without loss of generality, we may assume that \(a\) has no \(n\)th roots in \(k\), i.e.\ that \(X^n - a\) is an irreducible polynomial. Hence, \(A\) is an extension of degree \(n\) over the base field \(k\) and therefore also a finite-dimensional \(k\)-algebra.

We remark that \(A\) is a \(k[C_n]\)-module. For an element \(h \in k[C_n]\), we have:
\[
    h \cdot \zeta = \left(\sum_{j} h_j \, g^{\, j}\right) \cdot \zeta = \sum_{j} h_j \, \zeta^j
\]
and we can extend this action linearly on all \(a \in A\).

Using a result from a previous homework, we can prove that \(k \subseteq A\) is \(H^*\) Hopf Galois by showing that the morphism
\begin{align*}
    \mu \colon A \tensor k\left[C_n\right] &\to \End_k (A) \\
    a \tensor h &\mapsto (b \mapsto a (h \cdot b))
\end{align*}
is bijective:
\begin{itemize}
    \item \(\mu\) is injective. Suppose that the maps \(b \mapsto a (h \cdot b)\) and \(b \mapsto a' (h' \cdot b)\) are equal. Then, for any fixed \(b \in A\), we must have \(a (h \cdot b) = a' (h' \cdot b)\). In particular, taking \(b = \zeta\) we get
    \begin{align*}
        a \sum h_j \, \zeta^j &= a' \sum h_j' \, \zeta^j \\
        \implies
        \sum a h_j \, \zeta^j &= \sum a' h_j' \, \zeta^j
    \end{align*}
    Since the \(\zeta^j\) form a basis of \(A = \rationals(\zeta)\), we deduce that \(a h_j = a' h_j'\) (in \(k\)) for every \(j\), whence \(ah = a' h'\) (in \(k [C_n]\)). Then
    \begin{gather*}
        ah = a'h' \implies h = \frac{a'}{a} h' \\
        \implies
        a \tensor h = a \tensor \frac{a'}{a} h' = \frac{a'}{a} a \tensor h' = a' \tensor h'
    \end{gather*}
    as claimed.

    \item \(\mu\) is surjective. Every \(k\)-linear endomorphism of \(A\) must map each \(\zeta^i\) into some \(\zeta^j\) (or possibly a \(k\)-multiple of it). If we know what happens for \(\zeta\), we know what happens for all \(\zeta^i\).
    
    Let \(M \in \End_k (A)\). Suppose that \(M(\zeta) = \lambda \zeta^i\). Then, setting \(h = \lambda c^i\), we get
    \[
        h \cdot \zeta = \lambda (c^i \cdot \zeta) = \lambda \zeta^i
    \]
    which shows that \(\mu(\lambda, c^i) = M\).
\end{itemize}
\end{proof}

\begin{exercise}
For \(k = \rationals \subseteq \adjoin{\sqrt[3]{2}} = E\), a separable extension that is not normal, show that there is no subgroup \(G \leq \Aut(E)\) such that \(E^{G} = k\). Thus \(k \subseteq E\) is not \(G\)-Galois (in the classical sense) for any group \(G\).
\end{exercise}
\begin{solution}
Let \(\sigma\) be an automorphism of \(E\). By the properties of field automorphisms:
\begin{itemize}
    \item It must map \(1_{\rationals}\) to \(1_{E}\), since it is a unitary ring homomorphism.
    \item It fixes \(\integers\), since \(\sigma(n) = \sigma(1 + \dots + 1) = \sigma(1) + \dots + \sigma(1) = n \sigma(1) = n\).
    \item It fixes \(\rationals\), since \(\sigma(a/b) = \sigma(a) \sigma\left(b^{-1}\right) = \sigma(a) \sigma(b)^{-1} = a/b\).
\end{itemize}
By the pigeonhole principle, it must send \(\sqrt[3]{2}\) to some \(a \sqrt[3]{2}\), with \(a \in \rationals\). Applying \(\sigma\) and raising to the third power, we get
\[
    2 = \sigma(2) = \sigma\left(\sqrt[3]{2}^3\right) = \left(\sigma\left(\sqrt[3]{2}\right)\right)^3 = \left(a \sqrt[3]{2}\right)^3 = a^3 \cdot \left(\sqrt[3]{2}\right)^3 = 2 a^3
\]
therefore \(a\) must be equal to \(1\). But this shows that \(\sigma\) is the identity.

Since \(\Aut(E)\) is the trivial group, the only possibility for the subgroup \(G\) is also the trivial group. But \(E^{G} = E\), meaning there is no subgroup of \(\Aut(E)\) whose fixed field is \(k\), showing that the extension \(k \subseteq E\) is not Galois in the classical sense.
\end{solution}
